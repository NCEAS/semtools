\documentclass[11pt,onecolumn]{article}

\usepackage{fullpage}

\newcommand{\owlifier}{\textsf{owlifier}}

\title{\textsf{Owlifier}: An Application for Creating Simple OWL
  Ontologies from Spreadsheet-Based Knowledge Descriptions}

\author{Shawn Bowers \and Mark Schildhauer \and Joshua Madin \and
  Mathew Jones}

\date{}

\begin{document}


\maketitle

\begin{abstract}
\end{abstract}

\section{Introduction}


\section{\textsf{Owlifier} Syntax}

An \owlifier\ table defines an OWL-DL \cite{owl} ontology through a
set of \emph{blocks}.  Each non-empty row in an \owlifier\ table
corresponds to a block. The type of the block is given in the first
column of the row. The types of blocks supported by \owlifier\ are as
follows.  It is assumed below that if any of the properties or
concepts used in a block are defined, i.e., are not imported from
another ontology, then they are added to the current ontology.


\paragraph{Import Block.} Import blocks assign namespace labels to
external ontologies. Each external ontology is imported into the
current ontology. We refer to the ontologies of import blocks as
\emph{imported ontologies}.  Using import blocks, concepts and
properties of imported ontologies can be used within other blocks of
the table.  An import block has the form
\begin{itemize}
\item[]
  \begin{tabular}{|l|l|l|}\hline
    \texttt{import} & $\textit{NS}$ & $\textit{URI}$ \\ \hline
  \end{tabular}
\end{itemize}
where $\textit{NS}$ is a namespace label and $\textit{URI}$ is an OWL
ontology URI. Concepts and properties from imported ontologies are
referenced by prefixing the namespace label $\textit{NS}$ to the
corresponding concept or property name in the normal way.


\paragraph{Concept Block.} Concept blocks specify concept and
subconcept relationships. A concept block has the form
\begin{itemize}
\item[] 
  \begin{tabular}{|l|l|l|l|l|}\hline
    \texttt{concept} & $C_1$ & $C_2$ & \dots & $C_n$ \\ \hline 
  \end{tabular} \hfill ($n \ge 1$)
\end{itemize}
where each concept $C_i$ is asserted in the current ontology to
subsume a concept $C_{i+1}$, for $1 \le i < n$. Each $C_i$ in a
concept block induces a DL axiom \[C_i \sqsubseteq C_{i+1}\] If both
$C_i$ and $C_{i+1}$ are imported concepts, we say that the block
defines an ``articulation'' (i.e., mapping) between them.


\paragraph{Synonym Block.} Synonym blocks define an equivalence
relationship between concepts.  A synonym block has the form
\begin{itemize}
\item[] 
  \begin{tabular}{|l|l|l|l|l|}\hline
    \texttt{synonym} & $C_1$ & $C_2$ & \dots & $C_n$ \\ \hline 
  \end{tabular} \hfill ($n \ge 2$)
\end{itemize}
where each concept $C_i$ is asserted as being equivalent to concept
$C_{i+1}$ in the current ontology, for $1 \le i < n$. 


\paragraph{Overlap Block.} Except in certain situations (described
further below), defined concepts are assumed to be disjoint.  Overlap
blocks explicitly relax this assumption for a given set of
concepts. An overlap block has the form
\begin{itemize}
\item[] 
  \begin{tabular}{|l|l|l|l|l|}\hline
    \texttt{overlap} & $C_1$ & $C_2$ & \dots & $C_n$ \\ \hline 
  \end{tabular} \hfill ($n \ge 2$)
\end{itemize}
where each concept $C_i$ is allowed to share instances with each
concept $C_j$, for $1 \le i,j \le n$. In particular, $C_i$ and $C_j$
are not defined to be disjoint concepts in the current ontology. 

\paragraph{Property Block.} Property blocks define the required
\emph{object} properties of concepts. A property block has the form
\begin{itemize}
\item[]
  \begin{tabular}{|l|l|l|l|l|l|}\hline \texttt{property} & $P$ & $C_1$
& $C_2$ & \dots & $C_n$ \\ \hline
\end{tabular} \hfill ($n \ge 2$)
\end{itemize}
where $P$ is an object property and each $C_i$ is a concept, for $1
\le i \le n$.  For every concept $C_i$, the property block induces the
DL axiom \[C_i \sqsubseteq \exists P . C_{i+1} \] stating that each
instance of $C_i$ has, amongst possibly other things, a relationship
through $P$ to some instance of $C_{i+1}$.  For example, the block
\begin{quote}\texttt{property} \texttt{hasPart} \texttt{Body}
  \texttt{Head} \texttt{Eye} \texttt{Retina}
\end{quote} states that a body has at least one head, a head has at
least one eye, and an eye has at least one retina.

\paragraph{Attribute Block.} Attribute blocks are used to define the
required \emph{datatype} properties of concepts. An attribute block
has the form
\begin{itemize}
\item[]
  \begin{tabular}{|l|l|l|l|l|l|l|}\hline \texttt{attribute} & $P$ & $D$ & $C_1$
    & $C_2$ & \dots & $C_n$ \\ \hline
\end{tabular} \hfill ($n \ge 1$)
\end{itemize}
where $P$ is a datatype property, each $C_i$ is a concept for $1 \le i
\le n$, and $D$ is a datatype (\texttt{anyValueType}, \texttt{string},
\texttt{int}, etc.). For every concept $C_i$, the property block
induces the DL axiom \[C_i \sqsubseteq (\exists P . D) \sqcap (\le 1 ~
P)\] stating that each instance of $C_i$ has, amongst possibly other
things, a relationship through $P$ to exactly one data value of type
$D$. 


\paragraph{Value Block.} Value blocks define required datatype
property \emph{constant values} for concepts. A value block has the
form
\begin{itemize}
\item[]
  \begin{tabular}{|l|l|l|l|l|l|l|}\hline \texttt{value} & $P$ & $V$ & $C_1$
    & $C_2$ & \dots & $C_n$ \\ \hline
  \end{tabular} \hfill ($n \ge 1$)
\end{itemize}
where $P$ is a datatype property, $C_i$ is a concept for $1 \le i \le
n$, and $V$ is a datatype value. For each concept $C_i$, the value
block induces the DL axiom \[C_i \sqsubseteq (V \in P)\] stating that
each instance of $C_i$ has a value $V$ for property $P$.  The value
restrictions stated by value blocks are often used for defining
so-called \emph{value partitions} \cite{co-ode}.

\paragraph{Inverse Block.} Inverse blocks state that two object
properties are inverses of each other. That is, for inverse properties
$P_1$ and $P_2$ and concept instances $O_1$ and $O_2$, if $P_1(O_1) =
O_2$, then $P_2(O_2) = O_1$.  An inverse block has the form
\begin{itemize}
\item[]
  \begin{tabular}{|l|l|l|}\hline \texttt{inverse} & $P_1$ & $P_2$
\\ \hline
  \end{tabular}
\end{itemize}
where $P_1$ and $P_2$ are object properties. 

\paragraph{Transitive Block.} Transitive blocks state that a property
is transitive. That is, if $P$ is transitive and a concept instance
$O_1$ is related to an instance $O_2$ by $P$, and $O_2$ is related to
an instance $O_3$ by $P$, then $O_1$ is also by definition related to
$O_3$ by $P$. A transitive block has the form
\begin{itemize}
\item[]
  \begin{tabular}{|l|l|l|l|}\hline \texttt{transitive} & $P_1$ &
    $\dots$ & $P_n$
\\ \hline
  \end{tabular}
\end{itemize}
where $P$ is an object property. 

[others?, e.g., associatiave

\paragraph{Minimum Block.} Minimum blocks state the minimum
number of properties $P$ an instance of a concept may have.
Minimum blocks have the form
\begin{itemize}
\item[]
  \begin{tabular}{|l|l|l|l|l|l|l|}\hline \texttt{minimum} & $P$ & $N$ & 
    $C_1$ & $C_2$ & $\dots$ & $C_m$
    \\ \hline
  \end{tabular}
\end{itemize}
where $N$ is the minimum number of properties $P$ that instances of
concept $C_1$ may have to instances of concept $C_2$, $C_2$ to $C_3$,
and so on. A cardinality block induces the DL axiom \[C_i \sqsubseteq
(\le N P.C_{i+1})\] stating that each instance of $C_i$ must be related to
at least $N$ unique instances of $C_{i+1}$ via $P$. For example, the blocks
\begin{quote}
\texttt{minimum} \texttt{hasPart} 1 \texttt{Body} \texttt{Head} \\
\texttt{minimum} \texttt{hasPart} 2 \texttt{Head} \texttt{Eye}
\end{quote}
states that a body has at least one head and at least two eyes.

\paragraph{Maximum Block.} Maximum blocks state the maximum
number of properties $P$ an instance of a concept may have.
Maximum blocks have the form
\begin{itemize}
\item[]
  \begin{tabular}{|l|l|l|l|l|l|l|}\hline \texttt{maximum} & $P$ & $N$ & 
    $C_1$ & $C_2$ & $\dots$ & $C_m$
    \\ \hline
  \end{tabular}
\end{itemize}
where $N$ is the maximum number of properties $P$ that instances of
concept $C_1$ may have to instances of concept $C_2$, $C_2$ to $C_3$,
and so on. A cardinality block induces the DL axiom \[C_i \sqsubseteq
(\ge N P.C_{i+1})\] stating that each instance of $C_i$ may be related to
at most $N$ unique instances of $C_{i+1}$ via $P$. For example, the blocks
\begin{quote}
\texttt{maximum} \texttt{hasPart} 1 \texttt{Body} \texttt{Head} \\
\texttt{maximum} \texttt{hasPart} 2 \texttt{Head} \texttt{Eye}
\end{quote}
states that a body has at least one head and at least two eyes.


\paragraph{Sufficient Block.} Sufficient blocks state that any
instance having a property $P$ to an instance of a concept $C_2$ is a
sufficient condition for being an instance of a concept $C_1$. A
sufficient block has the form
\begin{itemize}
\item[]
  \begin{tabular}{|l|l|l|l|}\hline \texttt{sufficient} & $C_1$ & $P$ & $C_2$
    \\ \hline
  \end{tabular}
\end{itemize}
where $C_1$ is the target concept (i.e., denoting the concept
definition), $P$ is the sufficient property, and $C_2$ is the
sufficient concept. A sufficient block induces the DL axiom \[C_1
\equiv \exists P.C_2\] Sufficient blocks provide a mechansism to
construct simple class definitions (i.e., classes defined precisely by
other classes), primarily for use with value partitions. [NOTE: these
should be anded together?]

\paragraph{Description Block.} Description blocks assign plain-text
definitions to concepts and properties. A description block has the
form
\begin{itemize}
\item[]
  \begin{tabular}{|l|l|l|}\hline \texttt{description} & $T$ & $S$
\\ \hline
  \end{tabular}
\end{itemize}
where $T$ is either a property or a concept and $S$ is a description
string.

\paragraph{Note Block.} Note blocks add comments to the current
ontology, and are ignored by \owlifier.  A note block has the form
\begin{itemize}
\item[]
  \begin{tabular}{|l|l|l|}\hline \texttt{note} & $S$
\\ \hline
  \end{tabular}
\end{itemize}
where $S$ is a comment string.

*** Say something about relaxing block syntax ... to make it easier to
specify ontologies. Also, allow blocks to be given in any order.

\section{\textsf{Owlifier} Reasoning}

e.g., Disjoint Concept Inference. Need to describe here when we say
two concepts are disjoint. Other inferences are now possible as well.

Errors:
\begin{itemize}
\item Blocks with syntactic errors
\item Inverse properties can be between at most two properties. For
  instance, inverse(P1,P2) and inverse(P1,P3) is not allowed.
\item At most one description is allowed per property or concept.
\item Property and concept names must be disjoint
\end{itemize}

Warnings: 
\begin{itemize}
\item Cyclic concept hierarchies
\item Re-definition of imported concepts (have to define what this means)
\item Introduction of an inconsistency (can we show that this will
  never happen in a fully defined ontology, i.e., one without imports)? 
\item ...
\end{itemize}

\section{\textsf{Owlifier} Examples}

\begin{itemize}
\item Simple example, no imports, no warnings
\item Extension example
\item Articulation example
\end{itemize}

\subsection{\textsf{Owlifier} for OBOE}

How it works with OBOE.


\section{\textsf{Owlifier} API}

Flags:
\begin{itemize}
\item Turn on/off consistency checking/validation
\item Output format (OWL/RDF, 
\item Output inferred axioms
\item Ontology URI to use
\end{itemize}


\section{Discussion}

Implementation, etc.


\bibliographystyle{abbrv}
\bibliography{main}


\end{document}



