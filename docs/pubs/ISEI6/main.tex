\documentclass[preprint,number]{elsarticle} 
% other options: preprint, review, 1p, 3p, 5p, 
% bib options: authoryear, number, longtitle,
% other options: times


\newcommand{\Owlifier}{\textsf{Owlifier}}
\newcommand{\owlifier}{\textsf{owlifier}}

\newcommand{\myblock}[1]{\vspace{12pt}\noindent\textbf{#1}}

\newcommand{\secref}[1]{Section~\ref{#1}}
\newcommand{\figref}[1]{Figure~\ref{#1}}

\title{Owlifier: Creating OWL-DL Ontologies from Simple
  Spreadsheet-Based Knowledge Descriptions\tnoteref{t1}}
%\tnotetext[t1]{This work supported in part by NSF grants ...}

\author[smb]{Shawn Bowers\corref{cor1}}
\ead{sbowers@ucdavis.edu}

\author[jsm]{Joshua S. Madin}
\ead{jmadin@bio.mq.edu.au}

\author[mps]{Mark P. Schildhauer}
\ead[url]{schild@nceas.ucsb.edu}

\cortext[cor1]{Corresponding author}

\address[smb]{UC Davis Genome Center}

\address[jsm]{Dept. of Biological Sciences, Macquarie University, Australia}

\address[mps]{National Center for Ecological Analysis and Synthesis,
  UC Santa Barbara}



\begin{document}

\begin{abstract}
  Discovery and integration of data is important in many ecological
  studies, especially those that concern broad-scale ecological
  questions. Data discovery and integration is often a difficult and
  time-consuming task for researchers, which is due in part to the use
  of informal, ambiguous, and sometimes inconsistent terms for
  describing data content.  Ontologies offer a solution to this
  problem by providing consistent definitions of ecological concepts
  that in turn can be used to annotate, relate, and search for data
  sets.  However, unlike in molecular biology or biomedicine, few
  ontology development efforts exist within ecology. Ontology
  development often requires considerable expertise in ontology
  languages and development tools, which is often a barrier for
  ontology creation in ecology. In this paper we describe an approach
  for ontology creation that allows ecologists to use common
  spreadsheet tools to describe different aspects of an ontology.  We
  present conventions for creating, relating, and constraining
  concepts through spreadsheets, and provide software tools for
  converting these ontologies into equivalent OWL-DL
  representations. We also consider inverse translations, i.e., to
  convert ontologies represented using OWL-DL into our spreadsheet
  format. Our approach allows large lists of terms to be easily
  related and organized into concept hierarchies, and generally
  provides a more intuitive and natural interface for ontology
  development by ecologists.
\end{abstract}

\maketitle

\section{Introduction}

Within the fields of molecular biology and biomedicine considerable
effort has gone into developing ontologies for improving data
discovery and integration
\cite{ashburner00:_gene_ontol,bard04:_ontol_in_biolog}. While similar
benefits can be obtained for ecological data, far fewer efforts exist
to develop broad and consistent terminologies within ecology
\cite{madin08:_advan_ecolog_resear_with_ontol,parr20:_data_sharin_in_ecolog_and_evolut}.
The use of formal ontologies can significantly enhance metadata
descriptions of ecological data. For instance, annotating data with
ontology terms can both help users interpret data as well as enable
advanced capabilities for data discovery and integration, e.g., by
exploiting subsumption and part-of hierarchies as well as more formal
constraints such as cardinality restrictions on properties and term
equivalence \cite{madin08:_advan_ecolog_resear_with_ontol}.

Efforts to engage scientists in the development of ontologies
typically leverage the W3C Web Ontology Language (OWL)
\cite{smith04:_owl_web_ontol_languag_guide} as a standard XML syntax
for representing and sharing ontologies. A key advantage of OWL is
that it is supported by a wide range of generic tools, including
editors
\cite{knublauch04:_editin_descr_logic_ontol_with,kalyanpur05:_swoop},
reasoning systems
\cite{sirin07:_pellet,tsarkov06:_fact_descr_logic_reason}, query
languages
\cite{prudhommeaux08:_sparq_query_languag_for_rdf,motik05:_query_answer_for_owl_dl_with_rules},
and storage technologies
\cite{carroll04:_jena,broekstra02:_sesam}. Most of these tools,
however, are primarily targeted at experts in knowledge engineering
and software development familiar with the underlying description
logic semantics of OWL-DL \cite{grau08:_owl}. This is especially true
with ontology editors (such as Protege, SWOOP, etc.), which allow for
very detailed ontology specifications, but at the same time require
considerable understanding of the underlying ontology formalisms and
syntax. We see the lack of suitable ontology editing tools for
scientists without expertise in knowledge representation as one of the
major barriers for more wide-scale adoption of ontologies in ecology.


This paper presents a novel approach for ontology creation that aims
at being more intuitive for ecologists and that can be used to rapidly
construct large ontologies for describing scientific data. Our
approach is to allow scientists to use common spreadsheet-based tools
to describe, in an intuitive way, different aspects of an ontology,
and then to take these descriptions and convert them into full-fledged
OWL ontologies using a software application called \owlifier. An
\owlifier\ spreadsheet consists of a set of \emph{blocks} that have a
predefined template structure for users to fill in. Each non-empty row
in an \owlifier\ table constitutes a block. Each block defines
different aspects of an ontology including ontology classes,
subclasses, synonyms, and properties.  We also provide blocks for
plain-text descriptions of classes and properties, and for referencing
one or more existing ontologies (e.g., to extend an existing ontology
or to define ontology articulations). Blocks can be sparse (inheriting
from previous blocks), which can further simplify the creation of
large ontologies.

While not as expressive as OWL-DL, our approach can produce ontology
structures that are essential for improved data discovery and
integration \cite{madin07:_ontol_for_descr_and_synth}. Just as
important, because spreadsheet tools are frequently used by ecologists
to store and analyze data, \owlifier\ can provide ecologists with a
familiar and accessible user interface for ontology creation. This
approach also leverages the ability of spreadsheet tools to organize
and manipulate tabular data, e.g., allowing users to rapidly construct
class hierarchies from long lists of keywords.  In this way, an
ecologist can easily construct (or import) a set of terms, and then
incrementally organize these into class hierarchies, properties, and
constraints. In initial experiments with ecologists and evolutionary
biologists studying trait data, we found that \owlifier\ enabled them
to quickly and easily comprehend and construct useful ontologies.

The rest of this paper is organized as follows. In
\secref{sec:owlifier} we describe the basic syntax and semantics of
\owlifier. We define blocks that support a large subset of OWL-DL and
that also generally follow the ontology creation guidelines defined in
\cite{rector04:_owl_pizzas}. We also simplify certain aspects of
ontology creation using OWL-DL, e.g., by assuming classes are disjoint
by default (unless specified otherwise) and by applying implicit
property restriction closures \cite{rector04:_owl_pizzas}. In
\secref{sec:characteristics} we describe additional characteristics of
\owlifier\ and discuss issues with respect to classification and
reasoning. In \secref{sec:implementation} we briefly describe the
\owlifier\ implementation, and conlcude in \secref{sec:conclusion}
with related and future work. In general, the goal of \owlifier\ is
not to support all constructs in OWL-DL, but instead to provide a
higher-level ontology syntax (via spreadsheet blocks) that is easy for
ecologists to use and understand while also providing the necessary
constructs for developing typical ecological ontologies. By compiling
\owlifier\ to OWL-DL, we also allow for experts to refine and extend
the ontology using more advanced ontology editing tools if necessary.



\section{The Syntax and Semantics of \Owlifier}
\label{sec:owlifier}

As described above, an \owlifier\ table defines an OWL-DL
\cite{smith04:_owl_web_ontol_languag_guide} ontology through a set of
\emph{blocks} representing one or more ontology definitions.  Each
non-empty row in an \owlifier\ table corresponds to a block. The type
of the block is given in the first column of the row. 
% This section describes in detail each type of block supported by
% \owlifier.
We assume that if any properties or classes used in a block are not
imported from another ontology, then they are to be added to the
ontology being specified by the \owlifier\ table (i.e., the
``current'' ontology). In general, we name blocks according to the
terms used in
\cite{bowers08:_concep_model_framew_for_expres,madin07:_ontol_for_descr_and_synth}
as opposed to the names used for corresponding constructs in
OWL-DL. This choice of block names helps to simplify terminology
(e.g., we use ``relation'' below instead of ``object property''),
allows \owlifier\ to easilty generate ontologies that extend the
observational model of
\cite{bowers08:_concep_model_framew_for_expres,madin07:_ontol_for_descr_and_synth},
and avoids confusion with established terms commonly used within
ecology (e.g., ``class''). 

\myblock{Import Blocks.} Import blocks assign namespace labels to
external ontologies. Each external ontology is imported into the
current ontology. We refer to the ontologies of import blocks as
\emph{imported ontologies}.  Using import blocks, classes and
properties of imported ontologies can be used within other blocks of
an \owlifier\ table.  Rows containing import blocks take the form
\begin{itemize}
\item[]
  \begin{tabular}{|l|l|l|}\hline
    \textsf{import} & $n$ & $u$ \\ \hline 
  \end{tabular} 
\end{itemize}
where $n$ is a namespace label and $u$ is an OWL ontology URI. Classes
and properties from imported ontologies are referenced by prefixing
the namespace label $n$ to the corresponding class or property name in
the normal way. As an example, the following block imports the SWEET
``Earth Realm'' ontology \cite{raskin:_seman_web_for_earth_and}
\begin{tabbing}
  ~~\textsf{import}  \textsf{sweet} 
  \textsf{http://sweet.jpl.nasa.gov/ontology/earthrealm.owl}
\end{tabbing}
With this import block the class denoting Marine Ecosystems (a class
defined in the SWEET ontology) can be referred to from within an
\owlifier\ table using the expression
\textsf{sweet:MarineEcosystem}. Because this class refers to a class
in another ontology, we refer to it as an \emph{imported class}.


\myblock{Entity Blocks.} Entity blocks are the primary blocks used to
define ontologies. An entity block introduces new OWL classes and
specifies subclass relationships. Imported classes may also be used
within entity blocks by prefixing class names with namespace labels
(as described above).  Rows containing entity blocks take the form
\begin{itemize}
\item[] 
  \begin{tabular}{|l|l|l|l|l|}\hline
    \textsf{entity} & $c_1$ & $c_2$ & \dots & $c_n$ \\ \hline 
  \end{tabular} \hfill ($n \ge 1$)
\end{itemize}
where each class $c_i$ is asserted in the current ontology to subsume
$c_{i+1}$, for $1 \le i < n$. That is, each $c_i$ in an entity block
induces the description logic axiom $c_{i+1} \sqsubseteq c_i$.  If
both $c_i$ and $c_{i+1}$ are imported classes, we say that the block
defines an ``articulation'' (i.e., mapping) between the two
classes. The following entity block defines a simple subclass
hierarchy.
\begin{tabbing}
  ~~\textsf{entity} \textsf{PhysicalFeature} 
    \textsf{AquaticPhysicalFeature} \textsf{River}
\end{tabbing}
This block states that Physical Feature, Aquatic Physical Feature, and
River are classes; River is a subclass of Aquatic Physical Feature;
and Aquatic Physical Feature is a subclass of Physical Feature. The
following entity block introduces a new class via an imported class.
\begin{tabbing}
  ~~\textsf{entity} \textsf{sweet:MarineEcosystem} 
    \textsf{IntertidalEcosystem}
\end{tabbing}
This block states that Intertidal Ecosystem is a subclass of the
Marine Ecosystem class imported from the SWEET ontology. Similarly,
assuming ``marine'' denotes an existing ontology of marine ecosystem
concepts, the following block defines a simple class articulation.
\begin{tabbing}
  ~~\textsf{entity} \textsf{sweet:MarineEcosystem} 
    \textsf{marine:DeapSeaEcosystem}
\end{tabbing}
This block states that the Deap Sea Ecosystem class of the marine
ontology is a subclass of the Marine Ecosystem class of the SWEET
ontology (thus defining a mapping between these two ontologies).


\myblock{Synonym Blocks.} Synonym blocks define equivalence
relationships between ontology classes.  Rows containing synonym
blocks take the form
\begin{itemize}
\item[] 
  \begin{tabular}{|l|l|l|l|l|}\hline
    \textsf{synonym} & $c_1$ & $c_2$ & \dots & $c_n$ \\ \hline 
  \end{tabular} \hfill ($n \ge 2$)
\end{itemize}
where each class $c_i$ is equivalent to class $c_{i+1}$ in the current
ontology, for $1 \le i < n$. That is, each $c_i$ in a synonym block
induces a description logic axiom of the form $c_i \equiv
c_{i+1}$. The following synonym block creates a simple equivalence
relationship.
\begin{tabbing}
  ~~\textsf{synonym} \textsf{Maize} \textsf{Corn}
\end{tabbing}
This block states that the Maize and Corn classes are synonyms
(equivalent classes). Similar to entity blocks, synonym blocks often
contain imported classes for extending existing ontologies or defining
ontology mappings.


\myblock{Overlap Blocks.} Except in certain situations (described
further in \secref{sec:characteristics}), classes are generally
assumed to be disjoint in \owlifier.  Overlap blocks explicitly relax
this assumption by stating that a given set of classes may have
overlapping instances. Rows containing overlap blocks take the form
\begin{itemize}
\item[] 
  \begin{tabular}{|l|l|l|l|l|}\hline
    \textsf{overlap} & $c_1$ & $c_2$ & \dots & $c_n$ \\ \hline 
  \end{tabular} \hfill ($n \ge 2$)
\end{itemize}
where each class $c_i$ is allowed to share instances with each class
$c_j$, for $1 \le i,j \le n$. That is, a given $c_i$ and $c_j$ in an
overlap block are not defined to be disjoint classes in the current
ontology. As an example, consider the following entity blocks that
define the classes Estuary, Lagoon, and Marsh as subclasses of
Ecological Habitats.
\begin{tabbing}
  ~~\textsf{entity} \textsf{EcologicalHabitat} \textsf{Estuary} \\ 
  ~~\textsf{entity} \textsf{EcologicalHabitat} \textsf{Lagoon} \\ 
  ~~\textsf{entity} \textsf{EcologicalHabitat} \textsf{Marsh} 
\end{tabbing}
Given only these blocks, \owlifier\ treats Estuary, Lagoon, and Marsh
as disjoint classes. To relax this assumption and allow, e.g., types
of Lagoons to also be types of Estuaries, we explicitly add the
following overlap block
\begin{tabbing}
  ~~\textsf{overlap} \textsf{Estuary} \textsf{Lagoon}
\end{tabbing}
In general, overlap blocks are rarely used but provide a mechanism to
override the default behavior of \owlifier\ in asserting disjoint
classes.

% \myblock{Property Blocks.} Property blocks define basic object property
% domain and range constraints. , where the domain of an
% object property denotes the set of individuals that can have the
% property, and the range denotes the set of individuals that can be
% related to domain individuals through the property. Rows containing
% relationship blocks take the form 
% \begin{itemize}
% \item[]
%   \begin{tabular}{|l|l|l|l|}\hline \textsf{property} & $p$ & $c_1$
%     & $c_2$ \\ \hline
%   \end{tabular} \hfill ($n \ge 2$)
% \end{itemize}
% where $p$ is the object property, class $c_1$ is the property domain,
% and class $c_2$ is the property range. 

\myblock{Relationship Blocks.} Relationship blocks define
\emph{required} class object properties.  An object property within
OWL is a property defined between two class instances. Rows containing
relationship blocks take the form
\begin{itemize}
\item[]
  \begin{tabular}{|l|l|l|l|l|l|}\hline \textsf{relationship} & $p$ & $c_1$
    & $c_2$ & \dots & $c_n$ \\ \hline
  \end{tabular} \hfill ($n \ge 2$)
\end{itemize}
where $p$ is an object property and each $c$ is a class.  For every
class $c_i$, the relationship block induces the description logic axiom $c_i
\sqsubseteq \exists p . c_{i+1}$ stating that each instance of $c_i$
is $p$-related to some instance of $c_{i+1}$, for $1 \le i < n$.  For
example, the following block states that instances of the class
California Voles live in Grassy Areas.
\begin{tabbing}
  ~~\textsf{relationship} \textsf{livesIn} \textsf{CaliforniaVole} 
    \textsf{GrassyArea}
\end{tabbing}
In some cases, a particular property can apply to a sequence of
classes, and for convenience, each such class can be specified in
\owlifier\ using a single block. For example, consider the following
block.
\begin{tabbing}
  ~~\textsf{relationship} \textsf{directlyBelow} \textsf{Hypolimnion} 
    \textsf{Thermocline} \textsf{Epilimnion} 
\end{tabbing}
This block states that, e.g., within a thermally stratified lake, the
Hypolimnion layer is directly below the Thermocline layer, and the
Thermocline layer is directly below the Epilimnion layer. 

\myblock{Transitive Blocks.} Transitive blocks are special cases of
relationship blocks where the object property is asserted to be
transitive. If a property $p$ is declared to be transitive, whenever
$p$ relates an individual $x$ to an individual $y$, and an
individual $y$ to an individual $z$, then $p$ is also assumed to
relate $x$ to $z$. Rows containing transitive blocks take the form
\begin{itemize}
\item[]
  \begin{tabular}{|l|l|l|l|l|l|}\hline \textsf{transitive} & $p$ & $c_1$
    & $c_2$ & \dots & $c_n$ \\ \hline
  \end{tabular} \hfill ($n \ge 2$)
\end{itemize}
where $p$ is an object property and each $c$ is a class.  The
following block is a simple example of a transitive relationship.
\begin{tabbing}
  ~~\textsf{transitive} \textsf{hasPart} \textsf{Body}
  \textsf{Head} \textsf{Eye} \textsf{Retina}
\end{tabbing} 
This block states that every instance of the class Body has a Head as
a part, every instance of the class Head has an Eye as a part, and
every instance of the class Eye has a Retina as a part. Moreover,
because the \textsf{hasPart} property above is defined to be
transitive, it is possible to infer that every instance of Body also
has an Eye and a Retina as a part through the inherited relationship
restrictions $\textsf{Body} \sqsubseteq \exists \textsf{hasPart}
. \textsf{Head}$, $\textsf{Head} \sqsubseteq \exists \textsf{hasPart}
. \textsf{Eye}$, and $\textsf{Eye} \sqsubseteq \exists
\textsf{hasPart} . \textsf{Retina}$.


\myblock{Cardinality Blocks.} Cardinality blocks are also similar to
relationship blocks.  We consider three types of cardinality blocks
for defining minimum, maximum, and exact cardinality restrictions. A
minimum block states the smallest number of properties $p$ to distinct
individuals that an individual of a class may have.  Rows containing
minimum blocks take the form
\begin{itemize}
\item[]
  \begin{tabular}{|l|l|l|l|l|l|l|}\hline \textsf{min} & $p$ & $m$ & 
    $c_1$ & $c_2$ & $\dots$ & $c_n$
    \\ \hline 
  \end{tabular} \hfill $(n \ge 2)$
\end{itemize}
where $m$ is the minimum number of properties $p$ that instances of
class $c_i$ must have to instances of concept $c_{i+1}$, for $1 \le i
< n$.  For each class $c_i$, a minimum cardinality block induces the
description logic axiom $c_{i} \sqsubseteq \; (\le m) \; p.c_{i+1}$
stating that each instance of $c_i$ must be $p$-related to at least
$m$ unique instances of $c_{i+1}$. The following two blocks
demonstrate simple minimum cardinality constraints.
\begin{tabbing}
  ~~\textsf{min} \textsf{hasPart} 1 \textsf{Body} \textsf{Head} 
  \textsf{Nose} \\
  ~~\textsf{min} \textsf{hasPart} 2 \textsf{Head} \textsf{Eye}
\end{tabbing}
The first block states that instances of the class Body have at least
one Head as a part, which in turn have at least one Nose as a
part.\footnote{Cardinality restrictions ensuring participation to at
  least one property are typically not given through minimum
  cardinality blocks since they are also implied by relationship
  blocks.} The second block states that instances of the class Head
have at least two Eyes as parts. 

A maximum block states the largest number of properties $p$ to
distinct individuals that an individual of a class may have.  Rows
containing maximum blocks take the form
\begin{itemize}
\item[]
  \begin{tabular}{|l|l|l|l|l|l|l|}\hline \textsf{max} & $p$ & $m$ & 
    $c_1$ & $c_2$ & $\dots$ & $c_n$
    \\ \hline
  \end{tabular} \hfill $(n \ge 2)$
\end{itemize}
where $m$ is the maximum number of properties $p$ that instances of
concept $c_i$ may have to instances of concept $c_{i+1}$, for $1 \le i
< n$.  For each class $c_i$, a maximum cardinality block induces the
description logic axiom $c_i \sqsubseteq \; (\ge m) \; p.c_{i+1}$
stating that each instance of $c_i$ may be $p$-related to at most $m$
unique instances of $c_{i+1}$. The following two blocks demonstrate
simple maximum cardinality constraints.
\begin{tabbing}
  ~~\textsf{max} \textsf{hasPart} 1 \textsf{Body} \textsf{Head} 
  \textsf{Nose} \\
  ~~\textsf{max} \textsf{hasPart} 2 \textsf{Head} \textsf{Eye}
\end{tabbing}
The first block states that instances of the class Body have at most
one Head as a part, which in turn has at most one Nose as a part. The
second block states that instances of the class Head have at most two
Eyes as parts. 

An exact block ensures both a minimum and maximum number $m$ of
properties $p$ to distinct individuals that an individual of a class
must have. Rows containing exact blocks take the form
\begin{itemize}
\item[]
  \begin{tabular}{|l|l|l|l|l|l|l|}\hline \textsf{exact} & $p$ & $m$ & 
    $c_1$ & $c_2$ & $\dots$ & $c_n$
    \\ \hline
  \end{tabular} \hfill $(n \ge 2)$
\end{itemize}
where $m$ is the number of properties $p$ that instances of concept
$c_i$ must have to instances of concept $c_{i+1}$, for $1 \le i < n$.
For each class $c_i$, an exact block induces the description logic
axiom $c_i \sqsubseteq \; (= m) \; p.c_{i+1}$ stating that each
instance of $c_i$ must be $p$-related to $m$ unique instances of
$c_{i+1}$.

\myblock{Inverse Blocks.} Inverse blocks state that two object
properties are inverses of each other. If $p_1$ and $p_2$ are defined
to be inverse properties, whenever $p_1$ relates an individual $x$
to an individual $y$ then $p_2$ (as the inverse of $p_1$) is assumed
to relate $y$ to $x$.  Rows containing inverse blocks take the
form
\begin{itemize}
\item[]
  \begin{tabular}{|l|l|l|}\hline \textsf{inverse} & $p_1$ & $p_2$
\\ \hline
  \end{tabular}
\end{itemize}
where $p_1$ and $p_2$ are object properties. A common example of
inverse properties are \textsf{hasPart} and \textsf{partOf}, i.e., if
an individual $x$ has an individual $y$ as a part, then $y$ is
by definition a part of $x$.


% \myblock{Attribute Blocks.} Attribute blocks define required
% \emph{datatype} properties of classes. Unlike object properties, a
% datatype property within OWL is a property defined between a class
% instance and a literal value (e.g., a string or integer). Rows
% containing attribute blocks take the form
% \begin{itemize}
% \item[]
%   \begin{tabular}{|l|l|l|l|l|l|l|}\hline \textsf{attribute} & $p$ & $c_1$
%     & $c_2$ & \dots & $c_n$ \\ \hline
%   \end{tabular} \hfill ($n \ge 1$)
% \end{itemize}
% where $p$ is a datatype property and $c_i$ is a class, for $1 \le i
% \le n$. For each concept $c_i$, the attribute block induces the
% description logic axiom $c_i \sqsubseteq \exists p$ stating that
% each instance of $c_i$ has, amongst possibly other things, a
% property $p$ with a literal value. Attributes are often useful for
% As an example, the following ...


% \myblock{Value Blocks.} Value blocks define required datatype property
% \emph{constant values} for concepts. A value block has the form
% \begin{itemize}
% \item[]
%   \begin{tabular}{|l|l|l|l|l|l|l|}\hline \textsf{value} & $P$ & $V$ & $C_1$
%     & $C_2$ & \dots & $C_n$ \\ \hline
%   \end{tabular} \hfill ($n \ge 1$)
% \end{itemize}
% where $P$ is a datatype property, $C_i$ is a concept for $1 \le i \le
% n$, and $V$ is a datatype value. For each concept $C_i$, the value
% block induces the DL axiom \[C_i \sqsubseteq (V \in P)\] stating that
% each instance of $C_i$ has a value $V$ for property $P$.  The value
% restrictions stated by value blocks are often used for defining
% so-called \emph{value partitions} \cite{co-ode}.


\myblock{Sufficient Blocks.} Sufficient blocks are similar to synonym
blocks in that they state equivalences between classes. A sufficient
block serves as a modifier to entity blocks, relationship blocks
(including transitive blocks), and cardinality blocks, modifying the
associated description logic axioms by using equivalence ($\equiv$) in
place of sublass ($\sqsubseteq$). For example the block
\begin{tabbing}
  ~~\textsf{sufficient} \textsf{relationship} \textsf{hasPart} \textsf{Mammal} \textsf{Hair} 
\end{tabbing}
induces the description logic axiom $\textsf{Mammal} \equiv \exists
\textsf{hasPart}.\textsf{Hair}$ stating that \emph{any} individual
that has Hair as a part is a Mammal. Additionally, we allow
relationship blocks within a sufficient block to be extended with
\textsf{not} to state that absense of the property is a defining
characteristic of the class. For example the following blocks
\begin{tabbing}
  ~~\textsf{sufficient} \textsf{entity} \textsf{Mammal} \textsf{Eutheria} \\
  ~~\textsf{sufficient} \textsf{relationship not} \textsf{hasPart} \textsf{Eutheria} \textsf{EpipubicBone} 
\end{tabbing}
induces the axiom $\textsf{Eutheria} \equiv \textsf{Mammal} \sqcap
\neg \exists \textsf{hasPart} . \textsf{EpipubicBone}$ stating that
any Mammal that does not have an Epipubic Bone as a part is a
Eutheria. Note also that multiple sufficient blocks for a particular
class result in a single axiom in which each constraint is combined
via conjunction ($\sqcap$).


\myblock{Comment Blocks.} Comment blocks provide a mechanisms to add
plain-text comments to \owlifier\ tables. A description block attaches
a plain-text comment to classes and properties. Rows containing
description blocks take the form
\begin{itemize}
\item[]
  \begin{tabular}{|l|l|l|}\hline \textsf{description} & $c$ or $p$ & $s$
    \\ \hline
  \end{tabular}
\end{itemize}
where the string $s$ is associated as a comment to the class $c$ or
property $p$. A note block allows attaches comments to the current
ontology. Rows containing note blocks take the form
\begin{itemize}
\item[]
  \begin{tabular}{|l|l|l|}\hline \textsf{note} & $s$
\\ \hline
  \end{tabular}
\end{itemize}
where $s$ is a comment string.


\section{Additional Features of \Owlifier}
\label{sec:characteristics}

In this section, we briefly describe some of the additional features
of \owlifier, specifically focusing on the use of disjoint classes,
property closures, \owlifier\ reasoning, and additional block syntax.
 
\subsection{Disjoint Class Inference}

OWL is based on the open world assumption, which can lead to a number
of ontology development ``pitfalls'' for those new to the language
\cite{smith04:_owl_web_ontol_languag_guide,rector04:_owl_pizzas}. One
example is in the creation of disjoint classes (another concerns
property closures, described below). In particular, unless explicitly
stated, distinct classes within an OWL-DL ontology are never assumed
to be disjoint. However, in many ontologies a large number of classes
are typically defined as being disjoint (e.g., sibling classes), and
stating these disjoint constraints is often time-consuming since each
pair of classes must be given an explicit disjoint assertion. Editors
such as Protege \cite{knublauch04:_editin_descr_logic_ontol_with}
provide shortcuts via the user interface to create specific sets of
disjoint assertions, e.g., by allowing a user to define all children
of a particular class as disjoint. In general, however, many users
expect such classes to be disjoint by default
\cite{rector04:_owl_pizzas} and this expectation often leads to
modeling errors.

Alternatively, the default assumption in \owlifier\ is that distinct
classes are disjoint. Specifically, as an \owlifier\ table is
converted to an OWL-DL ontology, the system analyzes the class
hierarchy structure and identifies pairs of classes that are: (1) not
related via subclass relations (either direct or indirect subclasses);
(2) not defined as synonyms; and (3) not explicitly defined to overlap
via an overlap block. Each such pair of classes is then asserted by
\owlifier\ in the resulting ontology as being disjoint. As described
in \cite{rector04:_owl_pizzas}, undeclared disjoint classes are a
common problem in ontology development using OWL-DL and often limit
the utility of reasoning systems (by limiting the inferences that can
be obtained). The approach employed in \owlifier\ for handling
disjoint classes makes the common expectations of users the default
case, which in general should lead to a more intuitive ontology
editing environment and an overall fewer number of modeling mistakes.


\subsection{Object Property Closure}

Another common mistake for new users of OWL-DL concerns the user of
universal rather than existential property restrictions
\cite{rector04:_owl_pizzas}. In particular, an expression such as
$\texttt{Body} \sqsubseteq \forall \textsf{hasPart}$

\subsection{Reasoning in \owlifier}

4. is reasonable, e.g., it is possible to define values, value types,
and untangling

\subsection{Sparse Blocks} 

1. non-ambiguous blocks (no block leads to an ambiguous DL axiom)

5. we allow sparse blocks, and blocks can be given in any order



\section{Implementation of \Owlifier}
\label{sec:implementation}

figure showing basics owlifier interaction, description.

current status of implementation (subset of the syntax described here), perform classification via pellet/jena, performs disjoint check
 
ongoing extensions: additional syntax; specify delimeter characters;
perform inverse translations (easy to perform, e.g., ...); perform
closure computation (again, easy). 


\section{Conclusion}
\label{sec:conclusion}


related work: similar to dave's work on ``global taxonomic
constraints'', which assume certain structures of hierarchies, e.g.,
disjoint siblings. protege approach/plugin

future work: the ongoing extensions described above, support seamless
conversion into and out of owlifier, e.g., if someone extends owlifier
ontology, re-apply extensions after new owlifier changes. Also,
support translation to oboe. Additional testing and evaulation.



%\bibliographystyle{elsarticle-num}
\bibliographystyle{abbrv}
\bibliography{main}


\end{document}

