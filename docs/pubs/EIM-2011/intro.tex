
\section{Introduction}
\label{sec-intro}

A major challenge in environmental information management concerns
providing effective approaches for the discovery and integration of
heteregenous data sets. For instance, locating and combining relevant
observational data are often critical and time-consuming steps for
researchers studying phenomena at broad spatial, temporal, and
biological scales
\cite{worm06:_impac_of_biodiv_loss_ocean_ecosy_servic,pennings05:_do,green05:_compl_in_ecolog_and_conser,sorokina09:_detec_inter_variab_inter_obser_ornit_data,jackson01:_histor_overf_and_recen_collap}. The
underlying data sets used within such studies frequently differ in
subtle and complex ways, due in part to the protocols used for data
collection, the types of observations made, and the experimental and
other contextual information associated with the data set. These
differences in turn can lead to structural and semantic heterogeneity
among data sets that make them hard to discover using current data
management approaches and require considerable manual effort by
researchers needing to combine data sets.

A number of recent efforts within the earth and environmental
informatics communities are adopting the notion of an observation as a
key modeling concept for enabling improved discovery and integration
of scientific data
\cite{om,fox09:_ontol,tarboton07:_cuahs_commun_obser_data_model,cushing07:_compon_based_end_user_datab,balhoff10:_phenex,bowers08}. These
approaches provide higher-level observational data models for
describing and representing observations and measurements found in
underlying datasets by defining common ``core'' concepts such as the
entities or features being observed, measurement units and protocols,
and context relationships between observations \cite{om,bowers08}.  A
major goal of these approaches is to enable interoperability and
uniform access to data by abstracting away the underlying
representation details that often exist across scientific data sets.

In this paper we describe extensions to the Ecological Metadata
Language (EML) \cite{Fegraus07} and suppporting tools for enabling
improved discovery and integration of ecological data sets. Our work
is based on the Extensible Observations Ontology (OBOE)
\cite{bowers08,madin07:_ontol_for_descr_and_synth}, which represents a
generic observational model implemented in OWL-DL for describing
domain-specific observation and measurement types. Our approach adds
additional metadata in the form of semantic annotations that link
attributes within datasets to OBOE terms for describing the implicit
observation and measurement types found within data sets. Semantic
annotations are executable in the sense that they can be used to
convert a data set into a collection of observation and measurement
instances, providing a more uniform representation for expressing
queries and performing integration.  To support the creation of
annotations, we have extended the Morpho metadata editor
\cite{metacat02:_manag_heter_ecolog_data_using_morph} with a
high-level user interface as well as the Metacat data catalog
\cite{berkley01:_metac} for storing and querying annotations through a
new Semantic Mediation API. This API can also be used to perform basic
data-level integration tasks using our prior work on the EML Data
Manager Library \cite{leinfelder10:_metad_driven_approac_to_loadin}.

The rest of this paper is organized as follows. \secref{sec:framework}
briefly describes the various components used within our approach
including the extensions we have developed for Morpho and Metacat to
support semantic annotation. \secref{sec:application} describes the
types of data discovery queries and integration services supported by
our framework. \secref{sec:related} briefly describes related work,
and we summarize our contributions in \secref{sec:summary}.


%\section{Preliminaries}

%Briefly describe EML, Metacat, and Morhpo. Briefly describe oboe and
%semantic annotations.



%%% Local Variables: 
%%% mode: latex
%%% TeX-master: "main"
%%% End: 
