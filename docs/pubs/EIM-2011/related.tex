
\section{Related Work}
\label{sec:related}

The need for uniform mechanisms to describe observational data has led
to many proposals for observational data models (e.g., \cite{om,
  tarboton07:_cuahs_commun_obser_data_model,netcdf}), and ontologies
(e.g.,
\cite{fox09:_ontol,sweet,sheth08:_seman_sensor_web,mungall110:_integ}).
The work here is complementary to these efforts by providing a
framework and extensions to popular metadata tools (namely, Metacat
\cite{} and Morpho \cite{}) for supporting a generic observational
model through relatively light-weight, formal annotations. Our
approach also supports the use of domain-specific ontologies for
describing data as well as high-level query support for discovering
data sets. Our extensions also support ... integration ... tasks that
are crucial for determining the relevance of data.

Data management systems are increasingly employing various forms of
annotations to help improve search (e.g.,
\cite{DBLP:conf/icde/GeertsKM06,Reeve05SemanticAnnotation,DBLP:conf/vldb/Bhagwat
  CTV04}). For example, MONDRIAN \cite{DBLP:conf/icde/GeertsKM06}
employs an annotation model and query operators to manipulate both
data and annotations.  However, users must be familiar with the
underlying data structures (schemas) to take advantage of these
operators, which is generally not feasible for observational data in
which data sets exhibit a high degree of structural and semantic
heterogeneity.  Some efforts have also been carried out for leveraging
annotations, e.g., for the discovery of domain-specific data
\cite{obsdatasearch09,DBLP:conf/icde/StoyanovichMR10}, however, these
approaches are largely based on keyword queries, and do not consider
structured searches. Our work differs in that we consider a highly
structured and generic model for annotations with the aim of providing
a uniform approach for issuing structured data-discovery searches. Our
work is closely aligned to traditional data integration approaches
(e.g., \cite{HalevyDataIntegration06,kolaitis05}), where a global
mediated schema is used to (physically or logically) merge the
structures of heterogeneous data sources using mapping constraints
among the source and target schemas. As such, the observational model
we employ in our framqework can be viewed as a (general-purpose)
mediation schema for observational data sets.  This schema can be
augmented with logic rules (as target constraints) and uses the
semantic annotations as mapping constraints. However, instead of users
specifying logic constraints directly, we provide a high-level
annotation language that simplifies the specification of mappings and
more naturally aligns with the observation model. In addition, our
work focuses on implementing practical approaches for rewriting and
optimizing queries (that can include aggregation and summarization
operators) over our annotation approach.  In particular, our goal is
to create a feasible, scalable, and deployable system for applying
these approaches for data discovery and exploratory data analysis
within existing scientific data repositories.



%%% Local Variables: 
%%% mode: latex
%%% TeX-master: "main"
%%% End: 
