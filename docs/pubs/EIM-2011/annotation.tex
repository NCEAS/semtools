
\section{Annotation}

Semantic Annotations describe the observation model contained in a data object. The Annotation serves as a template for constructing a fully fleshed OBOE model of the [usually tabular] data. Less formal descriptions of the data attributes are often included in traditional metadata formats like EML. while there is some overlap between these two mechanisms - particularly with respect to measurement standards or units - semantic concepts applied to each attribute have the benefit of placing the descriptive burden on the ontology from which the concepts are drawn. Because the Annotation represents a potentially subjective perspective on the data, they are stored independently; the former referencing the latter.

The Annotation structure largely mirrors the core classes in OBOE. Observations are composed of Measurements of a specific Entity and are represented by a collection of Characteristics (usually just one) collected using a defined Protocol and Standard (i.e. unit). 
Tablular data objects map each attribute to a Measurement such that a collection of attributes usually pertain to the Entity of a shared Observation. These Observtions can provide context for other Observations such that the structure of the observational data model and collection paradigm are formally represented in the Annotation.

%%% Local Variables: 
%%% mode: latex
%%% TeX-master: "main"
%%% End: 
