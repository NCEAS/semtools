
\begin{abstract}
  Effective discovery and integration of ecological data within data
  management systems requires rich semantic information that can
  describe and relate the types of information contained within
  disparate data sets. Within the Semtools project, we have developed
  approaches for expressing and representing semantic annotations of
  data sets for supplementing attribute and data-level metadata with
  terms drawn from domain-specific ontologies. Annotations provide a
  formal mechansism that can be used together with reasoning systems
  to enhance existing data discovery and integration approaches.  In
  this paper we describe extensions to the Ecological Metadata
  Language (EML) and associated tools for storing and using semantic
  annotations.  Specifically, we describe new user interface
  components implemented within the Morpho EML metadata editor for
  capturing user-supplied semantic annotations, extensions to the
  Metacat system for storing and accessing annotations and
  corresponding OWL-DL ontologies, and a new API within Metacat that
  uses annotation metadata to provide concept-based search and
  integration of data sets.
%   The Data Manager paper from the last EIM included semantic
%   foreshadowing in the 'future directions' section and this is a
%   logical continuation of that work. Previously, we saw that the DM
%   library enabled data integration tasks provided we knew exactly
%   which data attributes from exactly which datasets were "compatible"
%   with each other. Here "compatible" means not simply structurally
%   similar (e.g. decimal numbers) but also conceptually similar
%   (e.g. ocean water temperature in C). By including rich semantic
%   annotations we initially show simple integration capabilities where
%   the subject domain of our data corpus is relatively constrained
%   (i.e. SBC-LTER) and the observational models represented are
%   minimally variable (i.e. context excluded). Semantically-enahnced
%   query capabilities facilitate this 'smart integration' in multiple
%   ways.
%   First, the corpus is refined through compound concept-based
%   queries that exercise OBOE-compatible ontological subsumption
%   hierarchies to locate roughly equivalent attributes across
%   datasets. Additional query constraints are defined in relation to
%   the precise observational model formally expressed in OBOE. Finally,
%   the actual data values are interrogated so that only those that fall
%   in the desired range are returned. From this resultant subset of
%   disparate data objects we create a single synthetic data product
%   that represents a 'materialization' of the observations. Technical
%   discussion will include high-level descriptions of the
%   infrastructure and tools employed with focus on our development in
%   the rapidly evolving semantic frontier where scalability and best
%   practices are particularly germane and equally elusive.
\end{abstract}



%%% Local Variables: 
%%% mode: latex
%%% TeX-master: "main"
%%% End: 
