
\section{Discovery}

\subsection{Concept query}
The semantic query interface primarily supports locating datasets by how well their observational models match the given criteria. Query criteria largely mirror the structure of an Annotation in that combinations of Entity, Characteristic and Protocol are specified and optionally compounded when increased precision is sought. 
By leveraging the relationships defined and/or inferred from the ontology we are able to increase recall beyond what is possible for simple keyword-based searches (Berkely et al, 20??). Broad queries return direct matches and also n-depth subclass matches. The queries can be quickly honed when using this browsing paradigm which allows rapid exploration of the data repository without the oneous of defining complete observational queries \emph{de novo}. Measurement templates as defined in OBOE compatible ontologies enable a single concept to proxy its constituent parts, namely the valid Characteristics of certain Entities that can be measured with certain Protocols and Standards. This short-hand query generation highlights one of the compelling applications of using OBOE compatible ontologies and can also be leveraged when authoring Annotations.
Using compound semantic query criteria applies a fine-grained filter on the datasets that are returned. Results may include only those datasets that include measurements for a set of specific Characteristics of a particular Entity. Furthermore, a query can specify that those measurements come from precisely the \emph{same instance of that Entity} as belaboured by the authors of the OBOE specification (Schildhauer here somewhere??).

\subsection{Data query}
For even more precise recall, the OBOE model can be partially \emph{materialized} during the query stage after which a data range filter can be applied. Different techniques are available for merging the Annotation with the data that it describes, but for performance reasons a hybrid approach has been adopted in which premliminary search results from a concept-based query are collated and only that subset is materialized. The Data Manager Library is conveniently used to load raw data objects because our corpus is described using EML and the Annotation syntax - both of which inform the correct use of the Data Manager query features. For any data attributes that match the concept query criteria, we then verify that those attributes contain data values within the range specifed by the initial semantic data query. Figure XX illustrates a semantic data query in which datasets recording organismal mass is greater than 0.5 grams are returned.

%%% Local Variables: 
%%% mode: latex
%%% TeX-master: "main"
%%% End: 
