\documentclass[10pt]{article}
%\title{}
%\author{}
%\date{}
\usepackage{algorithm}
\usepackage{algorithmic}
\usepackage{a4}
\usepackage{color}
\renewcommand\floatpagefraction{0.99}
\renewcommand\topfraction{0.99}
\renewcommand\bottomfraction{0.99}
\renewcommand\textfraction{.05}
\setcounter{totalnumber}{5}

%\setlength{\parindent}{0pt}
\setlength\topmargin{0in}
\setlength\headheight{0in}
\setlength\headsep{0in}
\setlength\textheight{8.5in}
\setlength\textwidth{6.5in}
\setlength\oddsidemargin{0in}
\setlength\evensidemargin{0in}

\newtheorem{example}{Example}[section]
\newtheorem{definition}{Definition}[section]
\newcommand{\from}[2]{{\bf[{\sc from #1:} #2]}}

\title{Data integration}
\author{}
\begin{document}
\maketitle

\section{Materialize Database}

{\bf HP:}{Copy the materials from materiazeDB to here. Skip now.}

\section{Query rewriting}

What kinds of queries? 

The information that users can have: metadata about the table. 

In the scientific data, it is unrealistic for the users to know the
table structure to perform the query. 

IR-style query: 
\begin{itemize}
\item KQ1: Give me the datasets that contain species ``Picea rubens'' observations.
\item KQ2: give me the datasets that have measurements on ``area''
  characteristics. 
\item KQ3: Give me the datasets that contain species ``Picea rubens''
  observations in ``California''. 
\end{itemize}

IR-style summarization query: 
\begin{itemize}
\item SQ1: Give me the dataset that contains at least five distinct ``Picea rubens'' observations.
\item SQ2: Give me the datasets that have measurements with average ``area''
bigger than 5.0 square feet. 
\end{itemize}

Use the materialized database, we would have four tables for all the datasets: 
entity, observation, measurement, context. 
and 4 tables for entity types, observation types, measurement types,
and context relationships.

To answer KQ1: it's simple join between 

How to perform query rewriting? 
m1:

data1 --> table 1
data2 --> table 2 
When we put the data as data tables, we add a unique record id for
each row in the data. This is used to get the unique record id. 

Metadata
Table definition

Table column definition

Map annotation information 

\begin{example}
KQ1: <concept>: <cond>
Concept can be observation, measurement, characteristic, standard,
unit, etc. 

Use query rewriting method: 
For KQ1, give me the datasets that contain species ``Picea rubens''
observations.
concept = observation, cond is species = ``Picea rubens''; 

The first step is to find the dataset that contain all the metadata. 
From the annotation, find the {\em measurement type} with {\em
  Characteristic} species. 
Here, it is $m_3$. From the mapping, we find that the attribute in the
table is {\em spp}. 

The second step is to find the data file which really have this
value. 
For this one, we find the data table, and do a select on the table
content. 

Use materialized database: 
The first step is the same, after we know the data file, then we
search the table {\em measurement\_instance} with the {\em mvalue=``Picea rubens''}. 

In this example, the computation cost is to search the different data
tables. To reduce this computation cost, we can materialize just the
measurement values. (How to do this? ) need more details. 

Half materialized: how to perform query over it. 
\end{example}

\begin{example}
SQ1: Give me the dataset that contains at least five distinct ``Picea
rubens'' observations.

The first step is to find the datasets that contain the metadata. 
The second step is to perform a query over the materialized database. 
%select mlabel, mvalue, count(*) as cnt from measurement_instance
%group by mvalue,mlabel having count(*)>=2;

Using the query rewriting, 
First, we find the data tables that contain the needed concepts. 
In each table, we need to figure out the distinct observations. 
From $m_i$ $obstype_i$, all the other measurements that are of
$obstype_i$. 
Group by the key measurements of $obstype_i$ if distinct yes.
If this observation type does not have any context, then the key
measurements are all the measurements directly defined under this
observatioin type. Otherwise, i.e., this observation type has context,
then the key measurements are all the measurements in the context
chain. 
\end{example}

\begin{example}
KQ3: Give me the datasets that contain species ``Picea rubens''
  observations in ``California''. 

Species = ``Picea rubens'' and observation has context {\em in}
another observation which has measurement of value ``California''. 
\end{example}

Using materializedDB, we can answer the query this way
First, Species --> mlabel, observation type 

select mtype, olabel from measurement\_type  \\
where chracteristic = 'Species';  \\

For the olabel found, \\
select context\_olabel\\
from context\_relationship\\
where olabel = 'olabel'; 

context\_olabel\_set is a set containing all the observation type
labels in the context chain. 

SELECT record\_id\\
FROM observation\_instance oi, observation\_instance coi,
context\_instance ci,
measurement\_instance mi, measurement\_instance cmi \\
WHERE mi.oid=ci.oid AND ci.oid = oi.oid and  cmi.oid=coi.oid AND ci.context\_oid =
coi.oid and oi.otype = 'olabel' and coi.otype $\in$
context\_olabel\_set
AND mi.mvalue=`Picea rubens' and cmi.mvalue='California';\\

Use query rewriting, what I can do? 
It's the same till to getting context\_olabel\_set. 

Get the asked condition columns and the context columns, 

select record\_id from data\_tables
where condition column 1='value1' and (context column1 = 'context
value' or context column2 = `context value', etc.);


\begin{example}
SQ2: Give me the datasets that have measurements with average ``area''
bigger than 5.0 square feet. 
\end{example}


\section{Experiments}

Synthetic data generator. 

Algorithm to materialize DB. 
Report the time and space. 

Synthetic query generator. 

Test algorithm to perform query over materiazed DB.
Test algorithm to perform query-rewriting over materiazed DB.

Test algorithm for half-materialized data.

\end{document}

