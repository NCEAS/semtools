\section{Introduction}\label{sec:intro}
In this work, we study the query processing over scientific
observation and measurement data using OBOE
model\cite{DBLP:conf/er/BowersMS08}. OBOE model is a conceptual model
used to interpret observation and measurement data. 

\subsection{Background}

%Application background
In many scientific domains (e.g., ecology, hydrology, earth science,
geology), people collect observational data. Such data
record the observed value of some real world entity at some specific
place and time. E.g., ecologists studying relationship between the 
growth pattern and the treatments often need to record the tree
heights. The collected data reflect the fact the tree height of a
specific tree is 30.1in on May 1, 2009 and 30.3in on May 1, 2010. 

%Characteristics of scientific observational data: non-normalized
Almost all such scientific data do not follow database higher normal
forms. Generally, scientists have their way in interpreting their
data, but they are not ready for any normalization process. 
For example, Table \ref{tb:dataset} is a simplified but typical dataset collected by
a scientist who study the growth pattern of trees. 
Obviously, the ``plt'' of the first two rows is the same place, and
the ``plt'' of the last two rows is the same. 
Here, their area information is redundant. 
In the real application, many columns have redundant information. 

%HP: this one has the coordinates, it's troublesome to explain. 
%% \begin{table}[htb]
%% \begin{center}
%% \begin{tabular}{|l|l|l|l|l|}
%% \hline
%% code & no & ht & plt & area\\\hline
%% piru & 1 & 35.8 & $(34^{\circ}8'3N,118^{\circ}14'37W)$ & 4.0\\\hline
%% piru & 2 & 36.2 & $(34^{\circ}8'3N,118^{\circ}14'37W)$ & 4.0\\\hline
%% piru & 1 & 25.7 & $(36^{\circ}10'30N,115^{\circ}8'11W)$ & 3.0 \\\hline
%% %abba & 1 & 15.6 & $(36^{\circ}10'30N,115^{\circ}8'11W)$ & 3.0\\\hline
%% capo & 1 & 15.6 & $(36^{\circ}10'30N,115^{\circ}8'11W)$ & 3.0\\\hline
%% \end{tabular}
%% \end{center}
%% \caption{Dataset}
%% \label{tb:dataset}
%% \end{table}


\begin{table}[htb]
\begin{center}
\begin{tabular}{|l|l|l|l|l|}
\hline
code & no & ht & plt & area\\\hline
piru & 1 & 35.8 & California & 4.0\\\hline
piru & 2 & 36.2 & California & 4.0\\\hline
piru & 1 & 25.7 & Oregon & 3.0 \\\hline
%capo & 1 & 7.8 & Oregon & 3.0\\\hline
Oriental poppy& 1 & 7.8 & Oregon & 3.0\\\hline 
\end{tabular}
\end{center}
\caption{Dataset}
\label{tb:dataset}
\end{table}
%HP: California poppy info
%Oregon iris: http://en.wikipedia.org/wiki/List_of_native_Oregon_plants
%http://en.wikipedia.org/wiki/California_poppy


%Introduce major concepts
In scientific domains where people collect observation and measurement
data, there are several commonly used and widely recognized canonical
concepts(\cite{oboe, om}). 
In this paper, we refer these concepts as OM concepts. 
The canonical concepts include {\em observation}, {\em measurement},
{\em characteristic}, {\em standard}, {\em protocol} or ({\em procedure}). 
For example, {\bf add an example to illustrate these concepts. }


%Introduce annotation, Motivation to use annotation
Generally, every scientific dataset goes into the data repository with
some metadata, e.g., Dublin core metadata\cite{***}, Darwin core metadata\cite{***}. 
However, all these metadata are at the dataset level, they did not
provide enough information for the data {\em content} inside the
dataset. 
To better make use of the data content in the repository, more systems
are embracing the ideas of having metadata on the data content. 
E.g., Some systems \cite{tdar} provides a mechanism to collect the
column/attribute level metadata, some system \cite{semtools} uses
{\em annotation} to add more semantic information to the data
content. In this work, we follow the terminology in \cite{semtools} and
use the term {\em annotation} to distinguish the 
metadata on the data content from those on data objects at a coarser
granularity. 
%How to represent the annotation and the symbols used
We propose tools for scientists to provide annotations to scientific
observational data. 
\from{HP}{Put a screen-dump of annotation.}


%Queries to such data and challenges to answer such queries.
The internal data structure of such scientific data repository is
generally unknown to a user who wants to pose a query to such scientific data
repository and get some useful information. 
So, it is unrealistic for a him/her to formulate a query based on the 
underlying data structure of the data.  
The easiest query they can pose to such data repository are keyword-query. 
Such keyword style query works well for the dataset metadata
(e.g., contributor information, dataset description, etc) using the
current Information Retrieval (IR) techniques. However, it
is far from enough to be used to search the data content because the
data content need to be interpreted with semantics. 
It is important in the scientific domain to have typical
kinds of queries. One fact we can use is that the domain
scientists know well the widely used OM concepts as discussed above. 
So, naturally, when searching such scientific data, they can provide
the concept information to restrain their queries and to find data sets
related to such OM terminologies. 
Given the dataset in Table \ref{tb:dataset}. 
People may ask the following observation and measurement (OM) queries. 

\begin{example}\label{eg:query}
Simple and summarization queries: 
\begin{itemize}
\item $Q_1$: Give me the data sets that contain species ``Picea rubens'' observations.
%$Q_2$: Give me the data sets that have measurements on ``area''  characteristics. 
\item $Q_2$: Give me the data sets that contain species ``Picea rubens''
  observations in ``California''. 
\item $Q_3$: Give me the data sets that contain at least five distinct
  ``Picea rubens'' observations.
\item $Q_4$: Give me the data sets that have trees with average ``height''
  than 20.0 in ``California''. 
% Shawn: an example on summarize one measure with respect to another.
% This could take two forms: one where the context forms the groups
% (like by year), and another where the two measurements are within the
% same context (like finding correlations).
\end{itemize}
\end{example}
\from{HP}{Challenges to answer such queries: uniqueness}

\from{HP}{Current data integration effort}

\from{HP}{Our effort in querying observational data: introduce OBOE}

\from{HP}{Need re-organization. The description in the
  following several sections may be moved to here later.}

\subsection{Contribution and paper organization}
Contributions of this work:
\begin{itemize}
\item We formulate a canonical set of queries over observation and
  measurement scientific data repository. Such formal queries can
  formalize most OM concepts related searches. 
\item We propose three methods to evaluate such OM
  queries. \from{HP}{Elaborate the three methods.}
\end{itemize}

This paper is organized as follows. 
Section \ref{sec:relatedwork} reviews works that are related to this
research. Section \ref{sec:dataquery} formalizes the data model and the
queries that people are interested to ask.  