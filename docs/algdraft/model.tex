\documentclass{article}
\begin{document}

\section{How to evaluate the quality of a conceptual model}

This question is posted based on the need on evaluating which conceptual model 
(e.g., OBOE, O\&M, etc) is a good one, or is better than the other. 
Unfortunately, the literature \cite{DBLP:journals/dke/Moody05, ***} shows that
there is no well-formed standard way to evaluae the quality of a conceptual model. 
On the contrary, most of the conceptual models are evaluated in an {\em ad hoc} manner. 
However, this article highlights a major principle in evaluating a conceptual model: 
a conceptual model is valuable only it is used in practice. 

Following this principle, we consider several factors in building our final model 
(either OBOE or an extended OBOE) in SONET project. The several factors include:\\  
1. Metadata model how complex it is?\\
2. How easy users can use the CM?\\
3. How easy it is to perform operations on the model? \\

\section{Term correspondences}
\begin{table}[htb]
\begin{tabular}{|l|l|l|l|}
\hline 
ER model & OBOE model term & O\&M term &others\\\hliine
Entity & Observation & Feature of interest & \\\hline
Characteristic & Measurement & ... & \\\hline
               & Characteristic & Property&\\\hline
               & Standard & ? & \\\hline
\end{tabular}
\end{table}

In O\&M, an {\em Observation} is defined as an action. 
It is modeled as a {\em Feature}. 
{\em Feature of Interest} is a feature that a user is interested in. 
Each feature can have properties. The {\em Property} can be {\em observed property} and other non-observed property. 
The value of {\em observed property} is obtained by an observer (e.g., a person)'s observation action.  

Procedure, Result, 

It seems to me: \\
Measurement  -- Feature type\\
Characteristic -- Property type\\
Standard \& Value -- Result\\
Procedure/protocal -- Procedure\\

How to represent the observation and measurement relationship in O\&M? \\
how to represent the observation - observation relationship in O\&M?\\
Any limitation for 

Several questions in understanding O\&M, 
what is {\em feature} (type and instance), {\em property} (type and instance).
According to the definition  4.19 
property-type is a characteristic of one or more feature types. 
It can have value. Or It must have value? 
It can have one/multiple values?  

Several questions to ask:
\begin{itemize}
\item How O\&M represent the relationship between Observation and Measurement in OBOE (1:m)
\end{itemize}

 
\section{Use cases}
We have several use cases that we can test the different data models. \\
{\bf to come. }

Some notes for myself: \\
sms: semantic mediation system

\bibliographystyle{abbrv}
\bibliography{model.bib}

\end{document}
