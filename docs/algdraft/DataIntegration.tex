\documentclass[10pt]{article}
%\title{}
%\author{}
%\date{}
\usepackage{algorithm}
\usepackage{algorithmic}
\usepackage{a4}
\usepackage{color}
\renewcommand\floatpagefraction{0.99}
\renewcommand\topfraction{0.99}
\renewcommand\bottomfraction{0.99}
\renewcommand\textfraction{.05}
\setcounter{totalnumber}{5}

%\setlength{\parindent}{0pt}
\setlength\topmargin{0in}
\setlength\headheight{0in}
\setlength\headsep{0in}
\setlength\textheight{8.5in}
\setlength\textwidth{6.5in}
\setlength\oddsidemargin{0in}
\setlength\evensidemargin{0in}

\newtheorem{example}{Example}[section]
\newtheorem{definition}{Definition}[section]
\newcommand{\from}[2]{{\bf[{\sc from #1:} #2]}}

\title{Data integration}
\author{}
\begin{document}
\maketitle

\section{Materialize Database}

{\bf HP:}{Copy the materials from materiazeDB to here. Skip now.}

\section{Query rewriting}

What kinds of queries? 

The information that users can have: metadata about the table. 

In the scientific data, it is unrealistic for the users to know the
table structure to perform the query. 

IR-style query: 
\begin{itemize}
\item KQ1: Give me the datasets that contain species ``Picea rubens'' observations.
\item KQ2: give me the datasets that have measurements on ``area''
  characteristics. 
\item KQ3: Give me the datasets that contain species ``Picea rubens''
  observations in ``California''. 
\item KQ: Give me the records....
\end{itemize}

IR-style summarization query: 
\begin{itemize}
SQ1: Give me the dataset that contains at least five ``Picea rubens'' observations.
SQ2: Give me the datasets that have measurements with average ``area''
bigger than 5.0 square feet. 
SQ3: 
\end{itemize}

Use the materialized database, we would have four tables for all the datasets: 
entity, observation, measurement, context. 
and 4 tables for entity types, observation types, measurement types,
and context relationships.

To answer KQ1: it's simple join between 

\section{Experiments}

1. Materialize DB (time + space) \\
2. Query over materialized DB 
3. 

\end{document}

