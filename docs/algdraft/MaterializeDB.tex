\documentclass[10pt]{article}
%\title{}
%\author{}
%\date{}
\usepackage{algorithm}
\usepackage{algorithmic}
\usepackage{a4}
\usepackage{color}
\renewcommand\floatpagefraction{0.99}
\renewcommand\topfraction{0.99}
\renewcommand\bottomfraction{0.99}
\renewcommand\textfraction{.05}
\setcounter{totalnumber}{5}

%\setlength{\parindent}{0pt}
\setlength\topmargin{0in}
\setlength\headheight{0in}
\setlength\headsep{0in}
\setlength\textheight{8.5in}
\setlength\textwidth{6.5in}
\setlength\oddsidemargin{0in}
\setlength\evensidemargin{0in}

\newtheorem{example}{Example}[section]
\newtheorem{definition}{Definition}[section]
\newcommand{\from}[2]{{\bf[{\sc from #1:} #2]}}
\begin{document}

\section{Key constraints}
 
%motivation to have key constraints.
In real world, the dataset collected by an ecologist to describe
measurents and observations is not normalized. I.e., two different
rows in the table may have a big amount of ``redundant'' 'information
to describe an object (e.g., entity). 

As discussed in \cite{***} %ER paper, 
we can use the OBOE cenceptual model to capture the semantic
relationships from the columns to their real meaning. 
However, the mapping from the columns to the OBOE model still cannot
directly solve the problem of getting concise informatin (e.g., unique
entity instances, unique observation instances, etc.)

Given Table \ref{tb:pltarea}(a), it records the fact that four times
of dbh of some trees are taken. 
Very often, an ecologist may ask questions like this:  Give me the
average dbh for every different trees. 
First, we have four rows here for trees. 
But how many tree entity instances are here is a question. 

There are several cases to consider:
\begin{itemize}
\item Case 1: The naive extreme way to interpret the data is that each
  observation is from different tree entity. Then, we have {\bf four}
  tree entities. This may be too strict. People may say, well, I have
  some observations for the same entity. 
\item Case 2: The second naive extreme way is to interpret that different {\em spp} represent the different tree entity. 
That's obvious that {\em piru} is different from {\em abba}. 
With this constraint, we get {\bf two} tree entities. However, an
obvious fact is also true that one {\em piru} tree cannot at the
different plots. (It is another story if this tree was migrated from one plot to another.) 
\item Case 3: Following Case 2, people want to further limit that the
  same {\em spp} in the same {\em plt} should represent the same tree
  entity set.  
%To achieve this, we use {\em identifying yes}.
%   Now, it should return
%	\[(A, piru, 36), (B, piru, 33.2), (B, abba, 34)\]
\end{itemize}

%Q2: Why we need to distinguish the same entities using {\em key yes}
%and {\em distinct yes}) to identify the same observation? 
%For the example, we have one observation for splot {\em A}.
%What's the semantic purpose of this? What kind of query may need this? 
%E.g., how many spots in this dataset? 

\begin{table}[htb]
\begin{center}
\begin{tabular}{cc}
\begin{tabular}{|l|l|l|}
\hline
plt & spp & dbh\\\hline
A & piru & 35.8 \\\hline
A & piru & 36.2 \\\hline
B & piru &33.2 \\\hline
B&abba&34\\\hline
\end{tabular}
&
\begin{tabular}{|l|l|l|l|}
\hline
plt & area & spp & dbh\\\hline
A & 1.0 & piru & 35.8 \\\hline
A & 1.1 & piru & 36.2 \\\hline
B & 2.0 &piru &33.2 \\\hline
B& 2.0 & abba&34\\\hline
\end{tabular}\\
(a) & (b)
\end{tabular}
\end{center}
\vspace{-0.2in}
\caption{Dataset}
\label{tb:pltarea}
\end{table}


Obviously, to answer this question, we not only need to figure out
what are the observations and measurements, we also need proper
constraints to help us understand whether different observations are
from the same entity or not. 

To tackle this problem, we proposed to use {\em key yes}, {\em
  distinct yes} and {\em identifying yes} constraints. 

\from{HP} {Describe or define {\em key yes},  {\em distinct yes},  {\em identifying yes}}

\begin{definition}[Key measurement types]{
The key mensurement tytpes of an observation are the measurement types
that are used to distinguish one observation/entity instance from the
other. }
\end{definition}
In most simple case, when an observation type does not have context, 
the key measurement types are specified with ``key yes''. 

{\bf Key constraints}: in this report (paper),  we use the term {\em
  key constraints} to refer to the general constraints of the above
three types {\em key yes}, {\em distinct yes} and 
{\em identifying yes}. 

\section{Valid annotation constraints}

Based on the semantic meaning of these key constraints, we can see
that they have some correlations. 
Before we give any further examples to explain such correlations, we
list the correlation rules among them.

\begin{itemize}
\item {\bf Rule 1}: If an observation type is specified with {\em distinct yes}, all
  its measurement types are automatically marked with {\em key yes}. 
\item {\bf Rule 2}: If the context of an observation type is specified with {\em identifying
    yes}. This observation type MUST have some key measurement
  types. And the context observation type also MUST have key
  measurement types too. 
\end{itemize} 

The following example shows an example why Rule 1 is needed. 
Given the data in Table \ref{tb:pltarea}(b) and the following
annotation. 

\noindent
{\bf observation} ``o1''  \textcolor{blue}{distinct yes}\\
\verb|    |{\bf  entity} ``Plot''\\
\verb|    |{\bf measurement} ``m1'' \textcolor{blue}{key yes}\\
\verb|        | {\bf characteristic} ``EntityName'' \\
\verb|        | {\bf standard} ``Nominal''\\
\verb|    | {\bf measurement} ``m2''\\
\verb|        | {\bf characteristic} ``area'' \\
\verb|        | {\bf standard} ``sqft''\\
\verb|        |$\cdots$ 

According to the annotation, if two plots have the same value for
``EntityName'', they represent the same plot observation. 
Obviously, it has problem to interprete the data in Table
\ref{tb:pltarea} with this annotatioin. 
E.g., the first and second rows catch information about plot with
EntityName $A$. Accoring to the annotation, they should be the same
plot obervation. 
However, the data shows that this one plot has two different areas
$1.0$ and $1.1$, so, there is confusion here.

For this case, we can have several ways to make the annoation
consistent. 
The first alternative to tackle this confusion is to change the
annotation.  We can let $m_2$ automatically becomes key measurement type of
$o_1$. Then, every different ($m_1$, $m_2$) value pair denote a
different observation instance which come from the different entity
instance.
The second alternative is to get rid of {\em distinct yes} for $o_1$,
which means that $(A,1.0)$ and $(A,1.1)$ are different observation instances of the same entity instance
$A$.  


%Here are some very basic rules to restrict the correlations among these constraints. 
%\begin{itemize}
%\item ``Distinct yes'' on an observation type implies that all its measurements collectively form a key of this observation type. 
%In the above example, we can specify ``distinct yes'' on ``$o_1$''
%without {\em explicitly} denote ``key yes'' on the measurements $m_1$
%and $m_2$. 
%The semantic meaning is that ($m_1$, $m_2$) collectively form key measurement for $o_1$.  
%\item Logically, we can specify ``Distinct no'' on an observation type and specify ``key yes'' on all of its measurements.
%In this case, when two rows (for the observation type) have the same
%values for these key measurements, they represent the different
%observation instances. Such a case means that there may exist other
%``variables'' that can distinguish one observation instance from the
%other, but this dataset does not catch this variable ({\bf HP: is
%``variable'' the right word to use? }). So, in real application, we
%do not encourage such situation to appear. 
%\end{itemize}

In our materialization algorithm, we assume that the annotations on the dataset are valid, i.e., they comply with the rules among these constraints. 

\section{Data materialization algorithm}
The {\bf MaterializeDB} algorithm catches the {\em key}, {\em distinct} and {\em identifying} constraints in the annotation during the materialization process. 
The input of the algorithm is the $Dataset$ and the annotations $A.*$ on it. 
Each row in the input dataset represents the information related to one or more observations and their contexts.
The input $A.*$ represents the annotation information.
More specifically,
\begin{itemize}
\item A.MeasType = $\{\langle \underline{MeasTypeId}, ObsTypeId, CharType, StdType, ProtType, Precision, isKey \rangle\}$;
\item A.ObservationType = $\{\langle \underline{ObsTypeId}, EntTypeId, isDistinct \rangle\}$ \\
	Note that, we hide the ``AnnotId'' in this schema, which denotes the resource the annotation is on.
    We do not include this because the algorithm focuses on dealing with annotations on one resource.
    This way, we can simplify the description.
\item A.ContextType = $\{\langle \underline{ObsTypeId, ContextObsTypeId, RelType}, isIdentify\rangle\}$
\item A.Map = $\{\langle \underline{MeasTypeId, ResAttribute, Cond}, Val\rangle\}$
\end{itemize}

The output of the algorithm is a set of materialized tables represented in the OBOE model and denoted by $OBOE.*$. In detail,
\begin{itemize}
\item $OBOE.Observation=\{\langle \underline{ObsId}, ObsType, EntId \rangle\}$ keeps all the observation instances;
\item $OBOE.Measurement=\{\langle \underline{MeasId}, ObsId, MeasType, Val\rangle\}$ for all the measurement instances;
\item $OBOE.Entity=\{\langle \underline{EntId}, EntType\rangle\}$ for all the entity instances;
\item $OBOE.Context =\{\langle \underline{ObsId, ConextObsId, ContextType}\rangle\}$ for all the context instances;
\end{itemize}

Algorithm \ref{alg:matdb} shows the framework of our algorithm. 
In this algorithm, we maintain two intermediate index structures ($EntIdx$ and $ObsIdx$) to keep track of the distinct entity and observation instances. 
$ObsIdx$ is the index structure maintained for the observation instances 
whose types are specified with {\em distinct yes}. 
This index maintains the mapping from observation type and key values to its corresponding observation instance id. 
Obviously, only when an observation type is specified with {\em distinct yes}, (i.e., we want to keep track of the same observation instances) we need to maintain their instances in this index. 

The key values can be calculated in three different cases.
In the first case where only one measurement of an observation type is specified with ``key yes'', the key value is the value of this ``key'' measurement.
 In the second case, several measurement types are marked with ``key yes'', the key value is the combined instance value of these several measurement types. 
In the third case, some context of this object type is marked with ``identifying yes'', the key value is the combination of the instance values of its own key measurement types and the instance values of its context observation's key measurements. 

$Entidx$ is the index structure for tracking the distinct entities. In case that some measurement type(s) of an observation type is/are specified with ``key yes'', 
if different observation instances have the same value on these key measurements, semantically, we interpret that they are of the same entity. 
The key value for each entity instance is computed in the same way as that for an observation instance.

The procedure of this algorithm is very straightforward. 
It processes the dataset in a row-wise manner. 
In dealing with each row, five steps are involved. 
The first step generates orphan measurement instances which are not connected to any observation instances.
The second step groups these measurement instances according to their observation types.
Then, for different observation types, the third and fourth steps materialize entity instances and observation instances respectively by either creating new oned or return existing ones. 
The last step assigns the context relationship among the different observation instances. 



\noindent {\bf Analysis of MaterializeDB}. \\
{\bf Time}: As can be seen that  this algorihm scans the original data file in a row-by-row manner without revisiting the already seen rows. So, it is linear in the size of the dataset. 
In addition, we use $EntIdx$ and $ObsIdx$ to facilitate the checking of unique entity and observation instances. Let $m$ be the number of distinct keys, each checking could take O(log(m)) time. 
So, in total, the algorithm runs in O(n log(m)) where $m\ll n$ generally. \\
{\bf Space}: $EntIdx$ and $ObsIdx$ are the intermdedite structures that we use in the algorithm. Since they keep the distinct entity and observation key values. The space complexity is of $O(m)$. 

\begin{algorithm} [htb]
\caption{{\bf MaterializeDB} ($Dataset, A.*$)}
\label{alg:matdb}
{\small
\begin{algorithmic}
\STATE /* {\em Dataset}: [Input] in the form of a flat file */\\
\STATE /* {\em A }:  [Input] Annotations*/
\STATE
\STATE $ObsIdx =\emptyset$;  /* Keep index $\langle ObsTypeId, KeyVal\rangle \rightarrow ObsId$*/
\STATE $EntIdx = \emptyset$;  /* Keep index $\langle ObsTypeId, KeyVal\rangle \rightarrow EntId$*/
\STATE
\FOR { (each $Row\langle A_1, A_2, \cdots, A_n\rangle \in Dataset$)}
    \STATE /* Step 1: Define measurement instances */\\
            $MeasSet$ = CrtMeasurement($Row, A.*$);
    \STATE
    \STATE /* Step 2: Partition the measurement instances according to observation types*/\\
        $ObsType2MeasIdx$ = PartMeas($MeasSet, A.*$);  /*$ObsType2MeasIdx=\{ObsTypeId \rightarrow \{mi\}\}$*/
    \STATE
    \STATE $ContextIdx = \emptyset$;  /* Keep index $ObsTypeId \rightarrow ObsId$ to materialize context*/
    \FOR{(each $ObsTypeId$ in $ObsType2MeasIdx$)}
      \STATE /* Step 3: Find or create the entity instance for each observation type partition */
        \STATE $EntId$ = MaterializeEntity($ObsTypeId, ObsType2MeasIdx, EntIdx, A.*, OBOE.*$);
        \STATE
        \STATE /* Step 4: Find or create the observation instance for each observation type partition of entity $EntId$ */
        \STATE MaterializeObs($ObsTypeId, EntId, ObsType2MeasIdx, ObsIdx, ContextIdx, A.*, OBOE.*$);
    \ENDFOR
    \STATE
    \STATE /* Step 5: Assign the context observation instances */  \\
            MaterializeContext($ContextIdx, A.*, OBOE.*$);
\ENDFOR
\RETURN OBOE;
\end{algorithmic}
}
\end{algorithm}

\begin{algorithm} [htb]
\caption{{\bf CrtMeasurement} ($Row, A.*$)}
{\small
\begin{algorithmic}
    \STATE/* Create new orphan measurement instances */
    \STATE $MeasSet = \emptyset$; /* Keep the set of new measurement instances */
	\FOR {(each $m = \langle MeasTypeId, ResAttribute, Cond, Val\rangle \in A.Map$)}
		
		\STATE {\bf if}{(($m.ResAttribute != Row.A_i.Attrname$) OR ($Row.A_i$ does not satisfy $m.Cond$))}
		  {\bf continue};
		\STATE $mi_{id}$ = GetNewMeasId($OBOE.Measurement$);
		\STATE {\bf if} ($m.Val!=NULL$) $MeasVal$ = $m.Val$;
		\STATE {\bf else} $MeasVal$ = $Row.A_i.Val$;
		\STATE Create a measurement instance $\langle mi_{id}, null, MeasType, MeasVal\rangle$ and add it to $MeasSet$;
	\ENDFOR
    \RETURN $MeasSet$;
\end{algorithmic}
}
\end{algorithm}

\begin{algorithm} [htb]
\caption{{\bf PartMeas} ($MeasSet, A.*$)}
{\small
\begin{algorithmic}
    \STATE /* Partition measurement instances according to their observation types*/
    \STATE $ObsType2MeasIdx=\emptyset$ /* Keep index for $ObsTypeId \rightarrow \{mi\}$ */
    \FOR{(each $mi \in MeasSet$)}
    \STATE $ObsTypeId$ = GetObsTypeId ($A.MeasType, mi.MeasTypeId$);
	\STATE Update $ObsType2MeasIdx$ by changing the item $ObsTypeId \rightarrow \{mi\}$;
    \ENDFOR
    \RETURN $ObsType2MeasIdx$;
\end{algorithmic}
}
\end{algorithm}

\begin{algorithm} [htb]
\caption{{\bf MaterializeEntity}($ObsTypeId, ObsType2MeasIdx, EntIdx, A.*, OBOE.*)$}
{\small
\begin{algorithmic}
\STATE $KeyVal$ = GetObsTypeKeys ($ObsTypeId, ObsType2MeasIdx$);
\STATE $HasKey$ = false;
\STATE {\bf if }{($ObsTypeId$ has key measurements OR is specified with distinct yes)}
     $HasKey$ = true;
\STATE $EntType$ = GetObsEntityType ($A.ObservationType, ObsTypeId$);
\STATE $CrtNewEntInst$ = true;
\IF{ ($HasKey$==true) }
	\STATE $EntId$ = GetEntId($ObsTypeId, KeyVal, EntIdx$); %/*From the current entity key index get the id of the entity instance*/
	\STATE {\bf  if} {($EntId!=NULL$)} $CrtNewEntInst$ = false;
\ENDIF
\IF{($CrtNewEntInst==true$)}
	\STATE $EntId$ = CrtEntId($EntType$);
	\STATE Create an entity instance $ei = \langle EntId, EntType\rangle $ and put $ei$ to $OBOE.Entity$;
	\STATE {\bf if} ($HasKey$==true) $EntIdx = EntIdx \cup \{\langle ObsTypeId, KeyVal \rangle \rightarrow EntId$\};
\ENDIF
\RETURN $EntId$;
\end{algorithmic}
}
\end{algorithm}

\begin{algorithm} [ht]
\caption{{\bf MaterializeObs}($ObsTypeId, EntId, ObsType2MeasIdx, ObsIdx, ContextIdx, A.*, OBOE.*$)}
{\small 
\begin{algorithmic}
    \STATE $KeyVal$ = GetObsTypeKeys ($ObsTypeId, ObsType2MeasIdx$);
    \STATE $IsObsDistinct$ = CheckIfObsDistinct($A.ObservationType, ObsTypeId$);
		\STATE $CrtNewObsInst$ = true;
	\IF{ ($IsObsDistinct$==true) }
		\STATE $ObsId$ = GetObsId($ObsTypeId, KeyVal, ObsIdx$);
		\STATE {\bf if}{($ObsId!=NULL$)}  $CrtNewObsInst$ = false;
	\ENDIF
	\IF{($CrtNewObsInst==true)$}
		\STATE Create an observation instance $oi = \langle ObsId,  EntId\rangle$ and put $oi$ to $OBOE.Observation$;
		\STATE {\bf  if} ($IsObsDistinct$==true) $ObsIdx = ObsIdx \cup \{\langle ObsTypeId, KeyVal \rangle \rightarrow ObsId$\};
           \ENDIF
    \STATE
    \STATE /* Maintain the measurement instances for this observation instance */
    \STATE $miSet$ = GetMeasInst($ObsType2MeasIdx, ObsTypeId$);
    \IF{($ObsId$ is a new one)} 
         \STATE Set the $obsId$ to each $mi \in miSet$ so that $mi$-s are not orphans;
	\STATE Put all the $mi \in miSet$ to $OBOE.Measurement$;
    \ELSE
         \STATE Discard all the $mi \in miSet$;
     \ENDIF

	\STATE $ContextIdx = ContextIdx \cup \{ObsTypeId \rightarrow ObsId\}$; /* $ContextIdx$ is also output*/

\end{algorithmic}
}
\end{algorithm}


\begin{algorithm} [htb]
\caption{{\bf MaterializeContext}($ContextIdx, A.*, OBOE.*$)}
{\small 
\begin{algorithmic}
\FOR{ ($ObsTypeId \rightarrow ObsId \in ContextIdx$)}
\STATE $ContextObsTypeId, Rel$ = GetContextObsTypeRel($A.ContextType, ObsTypeId$);
\IF{($ContextObsTypeId!= NULL$)}
	\STATE $ContextObsId$ = GetContextObsId($ContextIdx, ContextObsTypeId$);
	\STATE Create a context instance $ci=\langle ObsId, ContextObsId, Rel\rangle$;
	\STATE Put $ci$ to $OBOE.Context$;
\ENDIF
\ENDFOR
\end{algorithmic}
}
\end{algorithm}

\newpage
\begin{table}[htb]
\begin{center}
\begin{tabular}{|l|l|l|l|}
\hline
yr & spec & spp & dbh\\\hline
2007 & 1 & piru & 35.8 \\\hline
2008 & 1 & piru & 36.2 \\\hline
2008 & 2 & abba & 33.2 \\\hline
\end{tabular}
\end{center}
\vspace{-0.2in}
\caption{Dataset 1}
\label{tb:dataset1}
\end{table}



\begin{example} [Example with ``key yes'' and ``distinct yes'', without ``identifying yes'']
Take the data in Table \ref{tb:dataset1} \footnote{The detailed annotation is in page 6 of Shawn's powerpoint file.}
as an example to explain the algorithm. \\
For {\bf Row(2007, 1, piru, 35.8)}
\begin{itemize}
\item Step 1 creates four measurement instances: 
$\langle mi_1, null, Year, 2007\rangle$, $\langle mi_2, null, DBH, 35.8\rangle$, \\
$\langle mi_3, null, TaxonomicTypeName, Picea~rubens\rangle$, $\langle mi_4, null, EntityName, 1\rangle$,\\
and returns $MeasSet = \{mi_1, mi_2, mi_3, mi_4\}$;
\footnote{For all the instances, the measurement characteristic is set to represent Measurement Type}.

\item Step 2 returns $ObsType2MeasIdx = \{\{o_1\rightarrow \{mi_1\}, o_2 \rightarrow \{mi_2, mi_3, mi_4\}\}$. 
 \item Step 3-4: for  observation types $o_1$ and $o_2$,  materialize entity and observation instance
    \begin{itemize}
    \item for $o_1$ (with associated instance $mi_1$ of type $m_1$)
    \begin{itemize}
        \item Since $m_1$ is specified as key, get the $KeyVal=2007$;
        \item No entity with this key exists in $EntIdx$, create an entity $\langle ei_1, TemporalRange\rangle$; Now, $EntIdx=\{\langle o_1, 2007\rangle\rightarrow ei_1\}$.
        \item Since $o_1$ is specified as {\em distinct}, need to make sure we do not create redundant observation instances. No entry with the key $\langle o_1, 2007\rangle$ exists in $ObsIdx$, so, create an observation instance $oi_1$, which is of entity $ei_1$ and represented as $\langle oi_1, ei_1\rangle$. \\
	Now, $ObsIdx=\{\langle o_1, 2007 \rangle \rightarrow oi_1\}$
        \item Connect $mi_1$ to $oi_1$;
    \end{itemize}
    
    \item When deal with $o_2$,
        \begin{itemize}
        \item $KeyVal=1$.
        \item Create an entity instance $\langle ei_2, Tree\rangle$; $EntIdx=\{\langle o_1, 2007\rangle\rightarrow ei_1, \langle o_2, 1\rangle\rightarrow ei_2\}$.
        \item Create an observation instance $\langle oi_2, ei_2 \rangle$.
            No need to update $ObsIdx$ because $o_2$ is not identified as {\em distinct}.
        \item Connect $mi_2$, $mi_3$ and $mi_4$ to $oi_2$;
        \end{itemize}
    \end{itemize}
    \item Step 5 assigns the context relationship between $oi_1$ and $oi_2$;
    \end{itemize}
    
    
For {\bf Row (2008, 1, piru, 36.2)}
    \begin{itemize}
    \item Step 1 creates measurement instances $\langle mi_5, null, Year, 2008\rangle$,
    $\langle mi_6, null, DBH, 36.2\rangle$,\\
    $\langle mi_7, null, TaxonomicTypeName, Picea~rubens\rangle$,
    $\langle mi_8, null, EntityName, 1\rangle$\\
    and returns $MeasSet = \{mi_5, mi_6, mi_7, mi_8\}$;
    \item Step 2 gets $ObsType2MeasIdx = \{\{o_1\rightarrow \{mi_5\}, o_2 \rightarrow \{mi_6, mi_7, mi_8\}\}$
    \item Step 3-4: for observation types $o_1$ and $o_2$ materialize entity and observation instance
        \begin{itemize}
            \item for $o_1$
            \begin{itemize}
            \item  $KeyVal=2008$;
            \item Create an entity instance $\langle ei_3, TemporalRange\rangle$; \\
                $EntIdx=\{\langle o_1, 2007 \rangle \rightarrow ei_1, \langle o_2, 1 \rangle \rightarrow ei_2, \langle o_1, 2008 \rangle \rightarrow ei_3\}$.
             \item Create an observation instance $\langle oi_3, ei_3 \rangle$;\\
              $ObsIdx=\{\langle o_1, 2007\rangle\rightarrow oi_1, \langle o_2, 1\rangle\rightarrow oi_2, \langle o_1, 2008\rangle\rightarrow oi_3\}$
		\item Connect $mi_5$ to $oi_3$; 
            \end{itemize}
            \item When deal with $o_2$,
            \begin{itemize}
             \item $KeyVal=1$.
            \item {\bf item $\langle o_2, 1 \rangle \rightarrow ei_2$ is already in $EntIdx$, so get the entity id $ei_2$. No need to create an entity.}
            \item Since $o_2$ is not specified with {\em distinct yet}, we NEED to create an observation $\langle oi_4, ei_2 \rangle$. No need to update $ObsIdx$.
		\item Connect $mi_6, mi_7, mi_8$ to $oi_4$; 
            \end{itemize}
        \end{itemize}
    \end{itemize}
For {\bf ROW (2008, 2, abba, 33.2)}
    \begin{itemize}
        \item For $o_1$'s measurement $2008$, 
	\begin{itemize}
	\item Since $\langle o_1, 2008 \rangle \rightarrow ei_3$ already exists in $EntIdx$, {\bf no need to create a new entity}.
        \item Since $o_1$ is specified with {\em distinct yes}, and $\langle o_1, 2008 \rangle \rightarrow oi_3 $ already exists in $ObsIdx$, 
	{\bf no need to create a new OBSERVATION} and no need to put the measurement instance for $2008$ into OBOE model.
	\end{itemize}
    \end{itemize}

\end{example}
 
    
\begin{table}[htb]
\begin{center}
\begin{tabular}{|l|l|l|}
\hline
plt & spp & dbh\\\hline
A & piru & 35.8 \\\hline
A & piru & 36.2 \\\hline
B & piru &33.2 \\\hline
\end{tabular}
\end{center}
\vspace{-0.2in}
\caption{Dataset 2}
\label{tb2}
\end{table}

\begin{example} [Example with identifying]\label{eg2}
Let us use the data in Table \ref{tb2} as an example. \footnote{The detailed annotation information is at page 8 in Shawn's powerpoint file.}
Before we go through the algorithm step by step,  we first note that  $o_1$ and $o_2$ have key measurements $m1$ and $m_2$ respectively. 
So, we need to maintain the distinct entity instances for both of these two observation types. 
In addition, $o_1$ is specified with {\em distinct yes} while $o_2$ is not. So, we need to maintain the distinct observation instances for $o_1$ but not for $o_2$. 

For the {\bf first row}, 
\begin{itemize}
\item The first step generates three measurement instances $MeasSet=\{\langle mi_1, null, EntityName, A\rangle$, \\
$\langle mi_2, null, TaxonomicTypeName, Picea~rubens\rangle$, $\langle mi_3, null, DBH, 35.8\rangle\}$.
\item The second step gets $ObsType2MeasIdx = \{\{o_1\rightarrow \{mi_1\}, o_2 \rightarrow \{mi_2, mi_3\}\}$.
\item For each observation type, create entity and observation instances.
\begin{itemize}
\item For $o_1$, the key value is $A$. Since there is no such a key in $EntIdx$, we create an entity $ei_1$ of type $Plot$.\\
	$EntIdx = \{\langle o_1, A\rangle \rightarrow ei_1\}$. \\
	We create an observation instance $oi_1$ whose entity is $ei_1$.\\
	$ObsIdx = \{\langle o_1, A\rangle \rightarrow oi_1\}$.\\
	Connect the measurement instance $mi_1$ to observation instances  $oi_1$,
\item For $o_2$, 
	the key value is $(A, Picea~rubens)$ since it has context $o_1$ with ``{\em identifying yes}''.\\
  	We create an entity instance $ei_2$ of type $Tree$. \\
	$EntIdx = \{\langle o_1, A\rangle \rightarrow ei_1, \langle o_2, (A, Picea~rubens) \rangle \rightarrow ei_2\}$.\\
	We  create an observation instance $oi_2$ whose entity is $ei_2$.\\
	Connect the measurement instances $mi_2$ and $mi_3$  to observation instance $oi_2$.
\end{itemize}
\item The last step for this row is to connect the observations using context relationship. For this instance, we connect $oi_1$ to $oi_2$ with context ``{\em Within}''.
\end{itemize}

For the {\bf second row}, 
\begin{itemize} 
\item The first step defines three measurement instances $MeasSet=\{\langle mi_4, null, EntityName, A\rangle$, \\
$\langle mi_5, null,TaxonomicTypeName, Picea~rubens\rangle, \langle mi_6, null, DBH, 36.2\rangle\}$.
\item The second step gets $ObsType2MeasIdx = \{\{o_1\rightarrow \{mi_4\}, o_2 \rightarrow \{mi_5, mi_6\}\}$.
\item For each observation type, create entity and observation instances.
\begin{itemize}
\item For $o_1$, the key value is $\langle o_1, A\rangle$, $EntIdx$ already has an item for it with entity instance $ei_1$. No need to create a new instance for it. \\
To create observation instance, since $o_1$ is specified with ``{\em distinct yes}'' and the key value is $\langle o_1,A\rangle$, 
which corresponds to an existing observatin instance $oi_1$. So, we do not need to create a new observation for it.\\
When we try to connect the measurement instance $mi_4$ to observation instance,  we realize that we did not create a new observation instance for type $o_1$. 
So its related measurement instance $mi_4$ can be discarded. 
\item For $o_2$, the new key value is $\langle o_2, (A, Picea~rubens)\rangle$ , which corresponds to $ei_2$ in $EntIdx$, so no need to create a new instance for it either. \\
To create observation instances, since no ``{\em distinct yes}'' is specified, we create a new observation instance $oi_3$ for it.
Then, we connct the measurement instances $\{m_5, m_6\}$ to observation instance $oi_3$.
\end{itemize}
\end{itemize}

When we process the {\bf third row}, we have a new key value $\langle o_1, B\rangle$ for $o_1$, thus we create a new entity instance for it.
For $o_2$, we have new key value $\langle o_2, (B,~Picea~rubens)\rangle$ and create a new entity instance for it. Similarly, we need to create new observation instances for both type.

\end{example}


\end{document}

