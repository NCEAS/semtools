\documentclass[10pt]{article}
%\title{}
%\author{}
%\date{}
\usepackage{algorithm}
\usepackage{algorithmic}
\usepackage{a4}
\usepackage{color}
\renewcommand\floatpagefraction{0.99}
\renewcommand\topfraction{0.99}
\renewcommand\bottomfraction{0.99}
\renewcommand\textfraction{.05}
\setcounter{totalnumber}{5}
%\setlength{\parindent}{0pt}
%\pdfpagewidth 8.5in
%\pdfpageheight 11in
\newtheorem{example}{Example}[section]
\newcommand{\from}[2]{{\bf[{\sc from #1:} #2]}}
\begin{document}

How to annotate the above dataset using ``key yes'', ``distinct yes''
and ``identifying yes''.


\begin{table}
\begin{center}
\begin{tabular}{|l|l|l|l|l|l|}
\hline
Plot & sub-plot & Tmnt & Sp & Ind &wt\\\hline
1 & A & X &Aus&1&10\\\hline
1 & A & C &Bus&1&20\\\hline
1 & B & X &Aus&3&10\\\hline
1 & B & C &Bus&4&10\\\hline
2 & A & X &Aus&1&20\\\hline
2 & A & C &Bus&4&10\\\hline
2 & B & X &Aus&5&20\\\hline
2 & B & C &Bus&4&10\\\hline
\end{tabular}
\end{center}
\caption{A dataset with more complex information}
\label{tb:complexdb}
\end{table}

\noindent  Given the dataset in Table \ref{tb:complexdb}, users have different situations to catch. 
\begin{itemize}
\item Requirement 1: {\em Plot} with label ``1'' should refer to the same one physical plot (i.e., the Plot in the first 4 row means the same thing); similarly, 
  {\em plot} with label ``2'' should refer to the second physical plot (i.e., the Plot in the last 4 row means the same thing). 
\begin{itemize} 
\item This can be captured in annotation by putting ``Distinct yes'' for observation type {\em Plot} and ``key yes'' for its measurement type {\em PlotLabel}. 
\end{itemize}
\item Requirement 2: {\em sub-plot}s with the same lable should refer to the same physical sub-plot if they are within the same plot; 
but the sub-plot with the same label with different {\em Plot} label are different sub-plots.
E.g., Row 1 \{Plot=1, sub-plot=A\} refers the same sub-plot as that in Row 2, but is different from the one in Row 5 \{Plot=2, sub-plot=A\}. 
\begin{itemize} 
\item This can be captured by putting ``Distinct yes'' for observation type {\em SubPlot} , ``key yes'' for its measurement type {\em SubPlotLabel}. We need to denote {\em Plot} is its context and with {\em identifying yes} specified on this context. 
\end{itemize}
\item Requirement 3: {\em Tmnt} with the same lable should refer to the same treatment process (So that we can aggregate on different treatment process, e.g., on ``X'' or on ``C''.)  But the treatment at different sub-plot should refer to different treatment. 
 \begin{itemize} 
\item The first requirement can  be captured by treating all the
  Treatment with value ``X'' as the different entity instances with
  the same type (TmntType). The second requirement can be captured by
  treating the treatments in different sub-plots as observations of
  the different entity instances.  I.e., treatments in row 1 and row 3
  are of but are different observation instances which are of
  different entity instances. 
\item {\em Tmnt} has the context {\em sub-plot}. 
We don't need to specify ``key yes'', ``distinct yes'', or ``identifying yes''. 
%At the first glance, to represent this, {\em TmntType} for {\em Tmnt} should be specified with ``key yes''. 
%It should have context {\em sub-plot} which is specified with ``identifying yes''. 
%\item After further analysis, {\bf  one question arises: 
%The key measurements for the treatment observation is different from the key measurement  of the entity treatment}. 
%After considering the context, the key measurements for the treatment observation are \{Plotlabel, SubPlotLabel, TmntType\}.  
%When two rows have the same value on these measurements, they represent the same observation instance.
%However, the key measurement for the treatment entity is just \{TmntType\}. When two rows have the same value on it, they represent %the same entity instance. 
%The {\em identifying} constraint can only capture the observation context. 
%{\bf This problem is more obvious when we analyze Case 5}. 
%\item Another different annotation may be applied to catch this semantic. E.g., treat the {\em treatments} in different rows as different entity instances. 
%This way, the observation type and the entity type have the same key measurement types \{Plotlabel, SubPlotLabel, TmntType\}. 
%However, this problem still exists for Case 4 and Case 5. 

Summarization questions: \\
(1) Give me the average weight of the individuals with treatment ``X''. 
How can this question be answered after the annotation and materialization? 
This need to be answered after we annotate {\em Sp}, {\em Ind}, and {\em wt}. 
\end{itemize}

\item Requirement 4:  {\em Sp} with the same name should refer to the same species (e.g., a bird named {\em Aus} flies from sub-plot (1,A) to (1,B).)  
But the {\em Sp} with the same name at different sub-plot should refer to different observations of a specie.
%\begin{itemize}
%\item At the first glance, to represent this, {\em SpName} for {\em Sp} should be specified with ``key yes''. 
%It should have context {\em sub-plot} which is specified with ``identifying yes''. 
%\item {\bf The same problem as Case 3: The key measurements for the {\em Sp} observation is different from the key measurement  of the entity {\em Sp}}.
%the key measurements for the species observation are \{Plotlabel, SubPlotLabel, SpName\}. 
%However, the key measurements for the species entity is just \{SpName\}. 
%\end{itemize} 

\item Requirement 5:  {\em Ind} with the same label and and the same species name should refer to the same species.
But the individual (with the same label and the same species name) at different sub-plot should refer to different species observations.
%\begin{itemize}
%\item {\bf The same problem as Case 4 and Case 5: The key measurements for the {\em Ind} observation is different from the key measurement  of the entity {\em Ind}.}
%the key measurements for the species observation are \{Plotlabel, SubPlotLabel, SpName, Ind\}. 
%However, the key measurement for the species entity is just \{SpName, Ind\}.  When two rows have the same value on these two columns, they represent the same entity instance. 
%In this case, the observation context of Ind is {\em Sp} and {\em Sub-plot}. But the entity context of Ind is just {\em Sp}. 
%\end{itemize} 
\end{itemize}

The annotation can be done as follows. 
\begin{itemize}
\item Observation type: Plot {\em distinct yes}
  \begin{itemize}
  \item Measurement type: PlotLabel {\em key yes}
  \end{itemize}
\item Observation type: SubPlot {\em distinct yes}
  \begin{itemize}
    \item Measurement type: SubPlotLabel {\em key yes}
      \item Context: Plot {\em identifying yes}. This means that the
        key measurement for SubPlot (SubPlotLabel) and the key
        measurement for Plot (PlotLabel) together form the key of
        SubPlot observation (and its related entity too.)
      \end{itemize}
\item Observation type: Treatment {\em distinct yes}
  \begin{itemize}
    \item EntityType: TreatmentProcedure 
    \item Measurement type: TmntType {\em key yes}
   \item Context: SubPlot {\em Identifying yes}
  \end{itemize}
\item Observation type: SpeciesIndividual
  \begin{itemize}
      \item Measurement type: Species, {\em key yes}
      \item Measurement type: IndLabel (represent {\em Individual}), {\em key yes} 
      \item Measurement type: Weight
      \item Context: SubPlot. Note: (1) if the users want to treat the
        individual (with the same lavel and the same species) at
        different sub-plots as different observation instances, here,
        {\em identifying yes} should be specified. Otherwise, (2) if the users want to treat the
        individual (with the same lavel and the same species) at
        different sub-plots as SAME observation instances, here, we
        don't need to specify {\em  identifying yes}.
   \end{itemize} 
\end{itemize}

%In summary, we can get a better idea about the problem described in
%the above use cases can when we answer the following two simple
%questions: \\
%Q1:  Will an entity type and an observation type (which is of the
%given entity type) always have the same key measurement type(s)?
%     The above use cases give situations that the answer is no.  \\
%Q2: is {\em identifying} itself enough to distinguish the key
%measurement for observation types and for entity types? 
%    My temporary answer to this question is no. 

%A general thinking: 
%the counterpart in RDB (Relational DataBase) is a relational scheme with key attributes. 
%Here, we have two levels of objects: entity level and instance level.
%Then, for different levels of objects, we need to have different ways to specify their key measurements. 

%\newpage




\end{document} 

